\documentclass[main.tex]{subfiles}
\begin{document}

\chapter{Contextvrije talen en hun gramatica}
\label{cha:contextvrije-talen}

\section{Contextvrije Grammatica}
\label{sec:contextvrije-grammatica}


\begin{de}
  Een \term{contextvrije grammatica} (\term{CFG}) is een $4$-tal $(V,\Sigma,R,S)$:
  \begin{itemize}
  \item $V$: een eindige verzameling niet-eindsymbolen. (variabelen of non-terminals)
  \item $\Sigma$: een eindig alfabet van eindsymbolen, disjunct met $V$. (terminals)
  \item $R$: een eindige verzameling regels (producties).\\
    Een regel is een koppel van \'e\'en niet-eindsymbool en een string van elementen uit $V \cup \Sigma_{\epsilon}$. We schrijven deze vaak met een $\rightarrow$ ertussen.
  \item $S\in V$: het startsymbool.
  \end{itemize}
\end{de}

\begin{de}
  Zij $c = (V,\Sigma,R,S)$ een CFG.
  Een string $f$ over $V \cup \Sigma_{\epsilon}$ wordt afgeleid uit een string $b$ over $V \cup \Sigma_{\epsilon}$ met behulp van $c$ als er een eindige rij strings $s_{0},\dotsc,s_{n}$ bestaat zodat het volgende geldt:
  \begin{itemize}
  \item $s_{0} = b$
  \item $s_{n} = f$
  \item $s_{i+1}$ wordt verkregen door in $s_{i}$ een niet-eindsymbool $x$ te vervangen door het overeenkomstige eindsymbool in een regel uit $R$.
    We noteren dit proces als volgt.
    \[ \forall i: s_{i} \leadsto s \]
  \end{itemize}
  \[ b \leadsto^{*} f \]
\end{de}

\begin{de}
  De taal $L_{c}$ bepaald door een CFG $c = (V,\Sigma,R,S)$ is de verzameling strings over $\Sigma$ die kunnen afgeleid worden van het startsymbool $S$.
  \[ L_{c} = \{ s \in \Sigma^{*}\ |\ S \leadsto^{*} s \}
\]
\end{de}

\begin{de}
  Een taal $L$ is \term{contextvrij} (een \term{CFL}) als er een CFG bestaat zodat die CFG die taal bepaalt.
  \[ \exists L:\ L = L_{CFG} \]
\end{de}

\begin{de}
  Zij $c = (V,\Sigma,R,S)$ een CFG en $s$ een string die afgeleid kan worden uit $S$.
  De string $s$ is \term{ambigu} als de meest-linke afleiding van $s$ uit $S$ niet uniek is. 
\end{de}

\begin{de}
  Twee contextvrije grammatica's $c_{1}$ en $c_{2}$ zijn \text{equivalent} als ze dezelfde taal bepalen.
  \[ L_{c_{1}} = L_{c_{2}} \]
\end{de}

\begin{st}
  De equivalentie van contextvrije gramatica's is een equivalentierelatie.

  \begin{proof}
    Inderdaad, de gelijkheid van talen is reflexief, transitief en symmetrisch.
  \end{proof}
\end{st}

\begin{de}
  We noemen een CFG $c = (V,\Sigma,R,S)$ \text{niet ambigu} als er in de taal bepaald door $c$ geen ambigue strings zitten.
\end{de}

\begin{de}
  Een taal noemen we \term{inherent ambigu} als er geen niet-ambigue CFG bestaat die deze taal bepaalt.
\end{de}

\begin{de}
  Een CFG $c = (V,\Sigma,R,S)$ heeft de \term{Chomsky Normaal Vorm} als elke regel \'e\'en vor de volgende vormen heeft:
  \begin{itemize}
  \item $A \rightarrow BC$ met $A$, $B$ en $C$ niet-eindsymbolen en $B$ en $C$ verschillend van $S$.
  \item $A \rightarrow a$ met $A$ een niet-eindsymbool en $a$ een eindsymbool.
  \item $S \rightarrow \epsilon$
  \end{itemize}
\end{de}

\begin{st}
  Voor elke CFG $c = (V,\Sigma,R,S)$ bestaat er een equivalente contextvrije grammatica in Chomsky Normaal Vorm.

  \begin{proof}
    Constructief bewijs.\\
    Kies een willekeurige CFG $c = (V,\Sigma,R,S)$, dan construeren we nu een equivalente CFG $c' = (V',\Sigma,R',S)$ in Chomsky Normaal Vorm.

    \begin{itemize}
    \item Om te beginnen zorgen we ervoor dat in $c'$ het startsymbool enkel links in een regel voorkomt.
      Vervang overal in de regels van $R$ $S$ door een nieuw niet-eindsymbool $X$ en voeg een regel $S\rightarrow X$ toe.
    \item Vervolgens zorgen we dat $\epsilon$ enkel rechtstreeks volgt uit $S$.

      Stel dat we twee regels $\mathcal{Q} = A \rightarrow \epsilon$ en $\mathcal{R} = B \rightarrow \gamma$ kunnen vinden in $R$ waarbij $A$ voorkomt in $\gamma$.
      Definieer een verzameling regels $V(\mathcal{Q},\mathcal{R})$ als de verzameling van de regels van de vorm $B\rightarrow \eta$ waarbij $\eta$ verkregen wordt door in $\gamma$ een combinatie van de voorkomens van $A$ te schrappen.
      Op die manier zijn alle mogelijkheden waarop $\epsilon$ kon voorkomen in de originele regel vervangen door een equivalente regel.

      Pas $c$ nu iteratief aan door elk paar regels $\mathcal{Q}$ en $\mathcal{R}$ te vinden en $V(\mathcal{Q},\mathcal{R})$ toe te voegen aan $R$.
      Deze iteratie is eindig omdat zowel elke opeenvolging van symbolen in een regel eindig is als het aantal regels.
      Verwijder vervolgens alle regels van de vorm $A\rightarrow \epsilon$
      
      De bekomen grammatica bepaalt nog steeds dezelfde taal.
      \clarify{waarom?}

    \item Nu zorgen we ervoor dat er steeds vooruitgang zit in de regels.
      Alle regels van de vorm $A\rightarrow B$ moeten dus weg.

      Stel dat we twee regels $\mathcal{Q} = A \rightarrow B$ en $\mathcal{R} = B \rightarrow \gamma$ kunnen vinden in $R$.
      Defineer een regel $U(\mathcal{Q},\mathcal{R}) = A \rightarrow \gamma$.

      Pas $c$ nu iteratief aan door elk paar regels $\mathcal{Q}$ en $\mathcal{R}$ te vinden en $U(\mathcal{Q},\mathcal{R})$ toe te voegen aan $R$.
      Deze iteratie is eindig omdat het aantal regels in $R$ eindig is.
      Verwijder vervolgens alle regels van de vorm $A\rightarrow B$.

      De bekomen grammatica bepaalt nog steeds dezelfde taal.
      \clarify{waarom?}

    \item Er zijn nu nog $4$ soorten regels te behandelen:
      \begin{itemize}
      \item $A\rightarrow BC$ met $B$ en $C$ twee niet-eindsymbolen.
        Deze laten we voor wat ze zijn.
      \item $A\rightarrow \gamma$ waarin $\gamma$ bestaat uit minstens twee symbolen.
        Vervang elk eindsymbool $a$ door een nieuw niet-eindsymbool $A_{a}$ en voeg de regel $A_{a} \rightarrow a$ toe.

        Deze procedure is eindig omdat zowel elke opeenvolging van symbolen in een regel eindig is als het aantal regels.

        De bekomen grammatica bepaalt nog steeds dezelfde taal.
        \clarify{waarom?}

      \item $S\rightarrow \epsilon$.
        Deze laten we voor wat ze zijn.
      \item $A\rightarrow c$ met $c$ een eindsymbool.
        Deze laten we ook voor wat ze zijn.
      \end{itemize}
    
    \item Tenslotte maken we nog komaf met de regels van de vorm $A \rightarrow X_{1}\dotsc X_{n}$ met $n > 2$.
      Vervang elke regel van deze vorm door $n$ nieuwe regels:
      \[ A \rightarrow X_{1}Y_{1} \]
      \[ \forall i:\ Y_{i} \rightarrow X_{i+1}Y_{i+1} \]

      Deze procedure is eindig omdat zowel elke opeenvolging van symbolen in een regel eindig is als het aantal regels.

      De bekomen grammatica bepaalt nog steeds dezelfde taal.
      \clarify{waarom?}


    \item Wat er van $R$ overblijft noemen we nu $R'$ en $V'$ is hoogstwaarschijnlijk ook uitgebreid.
    \end{itemize}
  \end{proof}
\end{st}

\begin{st}
  Een afleiding van een string van lengte $n>0$ uit het startsymbool van een contextvrije grammatica in Chomsky normaalvorm heeft lengte $2n-1$.
  \begin{proof}
    Om een string van lengte $n$ af te leiden moet er voor elk symbool $a$ een regel van de vorm $A \rightarrow a$ gevolgd worden.
    Dit zijn dus als $n$ stappen van de afleiding.
    Elk van die $A$ moeten bovendien verkregen zijn door $S$ te `splitsen' in twee.
    Om $S$ te `splitsen' in $n$ stukken door telkens in $2$ te `splitsen' zijn er $n-1$ `splitsingen' nodig.
    De lengte van de afleiding is dus $n + n - 1 = 2n-1$.
  \end{proof}
\end{st}

\begin{de}
  Een CFG $c = (V,\Sigma,R,S)$ heeft de \term{Greibach Normaal Vorm} als elke regel \'e\'en vor de volgende vormen heeft:
  \begin{itemize}
  \item $A \rightarrow aX$ met $A$ een niet-eindsymbool, $X$ een (mogelijk lege) sequentie van niet-eindsymbolen en $a$ een eindsymbool.
  \item $S \rightarrow \epsilon$
  \end{itemize}
\end{de}

\begin{st}
  Voor elke CFG $c = (V,\Sigma,R,S)$ bestaat er een equivalente contextvrije grammatica in Breibach Normaal Vorm.
 
\TODO{bewijs}
\end{st}

\begin{st}
  Een afleiding van een string van lengte $n>0$ uit het startsymbool van een contextvrije grammatica in Breibach normaalvorm heeft lengte $n$.
 
  \begin{proof}
    Om een string van lengte $n$ af te leiden moeten er $n$ eindsymbolen te voorschijn komen door het toepassen van de regels van de grammatica.
    In elke stap van de vorm $A \rightarrow aX$ komt er precies \'e\'en eindsymbool te voorschijn.
    Er zijn zo dus $n$ stappen nodig. Bijgevolg is de lengte van de afleiding van een string van lengte $n$ van lengte $n$.
  \end{proof}
\end{st}


\section{Push-down automaten}
\label{sec:push-down-automaten}

\begin{de}
  Een \term{push-down automaat} (\term{PDA}) is een $6$-tal $(Q,\Sigma,\Gamma,\delta,q_{s},F)$:
  \begin{itemize}
  \item $Q$: een eindige verzameling toestanden.
  \item $\Sigma$: een eindig input alfabet.
  \item $\Gamma$: een eindig stapelalfabet.
  \item $\delta:\ Q\times \Sigma_{\epsilon} \times \Gamma_{\epsilon} \rightarrow \mathcal{P}(Q\times \Gamma_{\epsilon})$: een overgangsfunctie.
  \item $q_{s}$: de starttoestand
  \item $F\subseteq Q$: een verzameling aanvaardbare eindtoestanden.
  \end{itemize}
\end{de}

\begin{de}
  Een string $s$ wordt aanvaardt door een PDA $p = (Q,\Sigma,\Gamma,\delta,q_{s},F)$ als $s$ kan worden opgesplitst in $m$ delen $w_{i}$ zodat er $m+1$ toestanden $q_{j}$ bestaan en $m+1$ stacks $stack_{k}$ zodat het volgende geldt:
  \begin{itemize}
  \item $stack_{0} = \epsilon$: De stack is leeg aan het begin.
  \item $q_{0} = q_{s}$: De automaat start in de begintoestand.
  \item $q_{m} \in F$: De laatste toestand is een aanvaardbare toestand.
  \item $(q_{i+1}, y) \in \delta(q_{i}, w_{i+1}, x)$ waarbij $stack_{i} = xt$ en $stack_{i+1} = yt$ gelden met $t\in \Gamma^{*}$: De overgangen gebeuren volgens $\delta$.
  \end{itemize}
\end{de}

\begin{de}
  De taal $L$ bepaald door een PDA bestaat uit alle strings die door de PDA aanvaard worden.
\end{de}

\begin{lem}
  \label{lem:cfg-naar-pda}
  Vanuit elke CFG kunnen we een PDA construeren die dezelfde taal bepaalt.
\TODO{bewijs}
\end{lem}

\begin{opm}
  Het inputalfabet van een PDA komt overeen met de eindsymbolen van een CFG.
  Het stapelalfabet komt overeen met de unie van de niet-eindsymbolen en de eindsymbolen.
\end{opm}

\begin{lem}
  \label{lem:pda-naar-cfg}
 Vanuit elke PDA kunnen we een CFG construeren die dezelfde taal bepaalt. 
\TODO{bewijs}
\end{lem}

\begin{gev}
  \label{taal-pda-contextvrij}
  Als een taal bepaald wordt door een PDA is die taal contextvrij.
\TODO{bewijs}
\end{gev}

\begin{gev}
  Elke reguliere taal is contextvrij.\\\\
  We kunnen dit op twee manieren bewijzen
  \begin{itemize}
  \item Via een PDA:
    \begin{proof}
      Kies een reguliere taal $L$. Er bestaat dan een NFA $n$ die $L$ bepaalt.\footnote{Zie gevolg \ref{gev:reguliere-taal-NFA}.}
      Beschouw nu de PDA $p$ die $n$ `uitbreidt' door zijn stapel te negeren.
      $p$ bepaalt nu dezelfde taal als $n$, dus $L$ is contextvrij.\footnote{Zie stelling \ref{taal-pda-contextvrij}.}
\clarify{ formeler? }
    \end{proof}
  \item Door rechtstreeks een CFG op te stellen voor een reguliere taal.
    \begin{proof}
      Kies een reguliere taal $L$ over een alfabet $\Sigma$, er bestaat dan een reguliere expressie die $L$ bepaalt.\footnote{Zie eigenschap \ref{ei:reguliere-taal-expressie}.}
      We construeren nu inductief een CFG $c = (V,\Sigma,R,S)$ die ook $L$ bepaalt voor elke mogelijke reguliere expressie.
      $V$ wordt steeds ruim genoeg (maar natuurlijk steeds eindig) gekozen zodat het alle symbolen bevat die in $R$ voorkomen.
      Voor elke vorm van een reguliere expressie construeren we de regels die recursief aan $R$ moeten toegevoegd worden.
      \[
      \begin{array}{|c|c|}
        \hline
        E                           & R\\
        \hline
        a \text{ met } a \in \Sigma & \{ S \rightarrow a \}\\
        \epsilon                    & \{ S \rightarrow \epsilon \} \\
        \phi                        & \emptyset\\
        (E_1E_2)                    & \{ \S \rightarrow S_{1}S_{2} \} \\
        (E)^{*}                      & \{ S \rightarrow \epsilon,\ S \rightarrow S_{1}S \}\\
        (E_1|E_2)                   & \{ S \rightarrow S_{1},\ S \rightarrow S_{2} \}\\
        \hline
      \end{array}
      \]
    \end{proof}
  \end{itemize}
\end{gev}

\begin{st}
  Elke PDA bepaalt een contextvrije taal en elke contextvrije taal wordt bepaald door een PDA.
 
  \begin{proof}
    Inderdaad!\footnote{Zie lemma \ref{lem:cfg-naar-pda}.}\footnote{Zie lemma \ref{lem:pda-naar-cfg}.}
  \end{proof}
\end{st}

\section{Pompend lemma voor contextvrije talen}

\begin{st}
  Voor een contextvrije taal $L$ bestaat een getal $p$ (de pomplengte) zodat elke string $s$ van $L$ met lengte minstens $p$ kan opgedeeld worden in   vijf stukken $u$, $v$, $x$, $y$ en $z$ uit $\Sigma^{*}$
  \[ s = uvxyz \]
  \begin{enumerate}
    \item $\forall i \in \mathbb{N}:\ uv^{i}xy^{i}z \in L$
    \item $|vy| > 0$
    \item $|vxy| < p$
  \end{enumerate}
  \TODO{bewijs p 72}
\end{st}

\section{Een algebra van contextvrije talen?}

\begin{de}
  De unie $c_{1} \cup c_{2}$ van twee CFG's $c_{1}$ en $c_{2}$ is de CFG die de unie van $L_{c_{1}}$ en $L_{c_{2}}$ bepaalt.
\end{de}

\begin{st}
  De unie $c_{1} \cup c_{2}$ van twee CFG's $c_{1}$ en $c_{2}$ valt steeds eenvoudig te construeren.
  
  \begin{proof}
    Noem $S_{1}$ en $S_{2}$ de startsymbolen van respectievelijk $c_{1}$ en $c_{2}$.
    De unie van $c_{1}$ en $c_{2}$ heeft nu als symbolen de unie van de symbolen van $c_{1}$ en $c_{2}$.
    De unie heeft als regels de unie van de regels van $c_{1}$ en $c_{2}$.
    Merk op dat de symbolen van $c_{2}$ eerst moeten hernoemd worden zodat ze niet overeenkomen met de symbolen van $c_{1}$
    Voeg nu een nieuw symbool $S$ toe alsook de regels $S\rightarrow S_{1}$ en $S\rightarrow S_{2}$ om de unie van $c_{1}$ en $c_{2}$ te vervolledigen.
  \end{proof}
\end{st}

\begin{de}
  De doosnede $c_{1} \cap c_{2}$ van twee CFG's $c_{1}$ en $c_{2}$ is de CFG die de doorsnede van $L_{c_{1}}$ en $L_{c_{2}}$ bepaalt.
\end{de}

\begin{st}
  De doorsnede van twee CFG's bestaat niet noodzakelijk.
  De doorsnede van twee contextvrije talen is immers niet steeds contextvrij.
  \TODO{bewijs p 74}
\end{st}

\begin{de}
  Het complement $\bar{c}$ van een CFG $c$ is de CFG die het complement $\bar{L_{c}}$ van $L_{c}$ bepaalt.
\end{de}

\begin{st}
  Het complement van een CFG bestaat niet noodzakelijk.
  Het complement van een contextvrije taal is immers niet steeds contextvrij.
  \TODO{bewijs p 74}
\end{st}

\question{is de doorsnede van een CFL en een reguliere taal een CFL?}

\section{Ambigu\"iteit en determinisme}

\begin{de}
  Een contextvrije taal $L$ noemen we \term{deterministisch} als er een deterministische push-down automaat bestaat die $L$ bepaalt.
\end{de}

\begin{de}
  We noteren de verzameling van deterministische contextvrije talen als $DCFL$.
\end{de}

\begin{st}
  Een taal $L$ is deterministisch als $L$ niet inherent ambigu is.
  \begin{proof}
    Inderdaad, een ambigue taal kan onmogelijk deterministisch zijn.
    \clarify{waarom?}
  \end{proof}
\end{st}

\begin{opm}
  Er zijn wel niet-ambigue talen die niet-deterministisch zijn.
\end{opm}


\section{Contextsensitieve Grammatica}

\begin{de}
  Een \term{contextsensitieve grammatica} (\term{CSG}) is een $4$-tal $(V,\Sigma,R,S)$:
  \begin{itemize}
  \item $V$: een eindige verzameling niet-eindsymbolen. (variabelen of non-terminals)
  \item $\Sigma$: een eindig alfabet van eindsymbolen, disjunct met $V$. (terminals)
  \item $R$: een eindige verzameling regels (producties).\\
    Een regel is een koppel van twee strings van symbolen uit $V \cup \Sigma_{\epsilon}$.
  \item $S\in V$: het startsymbool.
  \end{itemize}
\end{de}


\begin{de}
  Zij $c = (V,\Sigma,R,S)$ een CSG.
  Een string $f$ over $V \cup \Sigma_{\epsilon}$ wordt afgeleid uit een string $b$ over $V \cup \Sigma_{\epsilon}$ met behulp van $c$ als er een eindige rij strings $s_{0},\dotsc,s_{n}$ bestaat zodat het volgende geldt:
  \begin{itemize}
  \item $s_{0} = b$
  \item $s_{n} = f$
  \item $s_{i+1}$ wordt verkregen door in $s_{i}$ een substring $x$ te vervangen door een overeenkomstige substring $y$ van een regel uit $R$.
    We noteren dit proces als volgt.
    \[ \forall i: s_{i} \leadsto s \]
  \end{itemize}
  \[ b \leadsto^{*} f \]
\end{de}

\begin{de}
  De taal $L_{c}$ bepaald door een CSG $c = (V,\Sigma,R,S)$ is de verzameling strings over $\Sigma$ die kunnen afgeleid worden van het startsymbool $S$.
  \[ L_{c} = \{ s \in \Sigma^{*}\ |\ S \leadsto^{*} s \}
\]
\end{de}

\begin{de}
  Een taal $L$ is \term{contextsensitief} (een \term{CSL}) als er een CSG bestaat zodat die CSG die taal bepaalt.
  \[ \exists L:\ L = L_{CSG} \]
\end{de}

\begin{de}
  Een CSG $c = (V,\Sigma,R,S)$ heeft de \term{Kuroda Normaal Vorm} als elke regel \'e\'en van de volgende vormen heeft.
  Zij $A$, $B$, $C$ en $D$ niet-eindsymbolen en $a$ een eindsymbool.
  \begin{itemize}
  \item $AB \rightarrow CD$
  \item $A \rightarrow BC$
  \item $A \rightarrow B$
  \item $A \rightarrow a$
  \end{itemize}
\end{de}

\TODO{ een lineair begrensde automaat LBA }
\question{ p 80: laatste restrictie: welke? }



\end{document}