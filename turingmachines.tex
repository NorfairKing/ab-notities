\documentclass[main.tex]{subfiles}
\begin{document}

\chapter{Turingmachines}
\label{cha:turingmachines}


\begin{de}
  Een \term{Turingmachine} is een $7$-tal $(Q,\Sigma,\Gamma, \delta, q_{s}, q_{a}, q_{r})$:
  \begin{itemize}
  \item $Q$ is een eindige verzameling toestanden
  \item $\Sigma$ is een eindig invoeralfabet dat $\#$ niet bevat.
  \item $\Gamma$ is een eindig bandalfabet dat $\#$ bevat, alsook $\Sigma$.
  \item $q_{s}$ is de starttoestand.
  \item $q_{a}$ is de aanvaardende toestand.
  \item $q_{r}$ is de verwerpende toestand.
  \item $\delta$ is de totale overgangsfunctie.
  \[ \delta:\ Q \times \Gamma \rightarrow Q \times \Gamma \times \{L,R,S\} \]
  \end{itemize}
  Een Turingmachine beschikt over een strook geheugen die aan twee kanten onbegrensd is.
  Deze strook is ingedeeld in vakjes die precies \'e\'en symbool bevatten.
  De Turingmachine beschikt bovendien over een lees- en schrijfeenheid die zich over een vakje bevindt dat de machine bijgevolg kan lezen en waarnaar de machine kan schrijven.
  De Turingmachine bevindt zich op elk moment in precies \'e\'en toestand.
\end{de}

\begin{de}
  De \term{werking van een Turingmachine} definieren we aan de hand van een Turingmachine $M = (Q,\Sigma,\Gamma, \delta, q_{s}, q_{a}, q_{r})$ en een invoerstring $s$.
  \begin{itemize}
  \item \textbf{Initialisatie}\\
    De geheugenstrook wordt eerst ge\"initialiseerd zodat ze enkel het symbool $\#$ bevat in op elk vakje.
    Vervolgens wordt vanaf een willekeurig vakje opeenvolgend de invoerstring in de vakjes op de band gezet.
    De lees- en schrijfeenheid wordt op het vakje gezet waar de string begint.
    De machine bevindt zich aan het begin in de begintoestand $q_{s}$.
  \item \textbf{Uitvoering}\\
    De Turingmachine $M$ voert achtereenvolgens de volgende stappen uit.
    \begin{itemize}
    \item De machine leest het symbool dat onder de lees- en schrijfeenheid staat.
    \item Op basis van de staat $q$ waarin de machine zich bevindt en het symbool $a$ onder de eenheid beslist de machine wat hij zal doen aan de hand van de overgangsfunctie.
      \[ \delta(q,a) = (q',a',r) \]
      De machine zal het symbool $a'$ schrijven, naar staat $q'$ overgaan en zicht in de richting $r$ verplaatsen.
    \end{itemize}
  \item \textbf{Beslissing}\\
    Zodra de machine zich ofwel in $q_{a}$ ofwel in $q_{r}$ bevindt stopt hij.
    Als de machine zich op het einde in $q_{a}$ bevindt zeggen we dat $M$ de string $s$ aanvaardt.
    Als de machine zich op het einde in $q_{r}$ bevindt zeggen we dat $M$ de string $s$ verwerpt.
    Merk op dat het ook kan voorvallen dat de machine bij een gegeven string $s$ niet ophoudt met werken.
  \end{itemize}
\end{de}

\begin{de}
  Elke turingmachine $M$ bepaalt een verdeling van $\Sigma^{*}$ in drie stukken:
  \begin{itemize}
  \item De strings die door $M$ worden geaccepteerd: $L_{M}$.
  \item De strings waarvoor $M$ in een oneindige lus belandt: $\infty_{M}$.
  \item De strings die door $M$ worden verworpen.
  \end{itemize}
\end{de}

\begin{de}
  Een turingmachine $M$ \term{herkent} $L_{M}$.
\end{de}

\begin{de}
  Een taal $L$ is \term{Turing-herkenbaar} als er een Turingmachine $M$ bestaat die $L$ herkent.
\end{de}

\begin{de}
  Een turingmachine $M$ \term{beslist} een taal $L$ als $M$ $L$ herkent en $\infty_{M}$ leeg is.
\end{de}

\begin{de}
  Een taal $L$ is \term{Turing-beslisbaar} als er een Turingmachine $M$ bestaat die $L$ beslist.
\end{de}

\begin{de}
  Een taal $L$ is \term{co-herkenbaar} als $\bar{L}$ herkenbaar is.
\end{de}

\begin{de}
  Een taal $L$ is \term{co-beslisbaar} als $\bar{L}$ beslisbaar is.
\end{de}

\begin{st}
  \label{st:herk-plus-coherk-is-besl}
  Als een taal $L$ zowel herkenbaar als co-herkenbaar is, dan is $L$ beslisbaar (en co-beslisbaar).
  \begin{proof}
    Kies een willekeurige herkenbare en co-herkenbare taal $L$.
    Er bestaat dan een Turingmachine $m$ die $L$ herkent en een Turingmachine $m'$ die $\bar{L}$ herkent.
    Construeer vanuit $m$ en $m'$ nu een machine $M$ die $L$ beslist.
    $M$ bestaat erin $m$ en $m'$ simultaan te laten lopen op de invoerstring $s$
    Zodra $m$ $s$ aanvaardt, aanvaardt $m$ $s$ en zodra $m'$ $s$ aanvaardt, verwerpt $m$ $s$.
    Voor elke mogelijke invoerstring stopt minstens \'enen van de machines $m$ en $m'$.
    $M$ zal dus altijd stoppen.
  \end{proof}
\end{st}

\begin{st}
  Er bestaat een taal die niet herkenbaar is.
  \begin{proof}
    Het aantal Turingmachines is aftelbaar oneindig (, maar enkel op isomorfisme na).
    Elke (equivalentieklasse van) Turingmachine(s) herkent bovendien precies \'e\'en taal.
    Het aantal talen over een gegeven alfabet is overaftelbaar oneindig.\footnote{Zie stelling \ref{st:overaftelbaar-veel-talen}.}
    Er bestaat dus minstens \'e\'en taal waarvoor er geen Turingmachine bestaat die die taal herkent.
    Die taal is niet herkenbaar.
    Sterker nog: ``De meeste\footnote{``De meeste'' heeft een heel specifieke wiskundige betekenis.} talen zijn niet herkenbaar.''
  \end{proof}
\end{st}


\begin{de}
  Een Turingmachine kan een andere Turingmachine als subroutine oproepen.
\extra{Hoe formaliseren we dit?}
\end{de}

\begin{st}
  Het aantal Turing-machines is aftelbaar oneindig (op isomorfisme na).

  \begin{proof}
    Een Turing-machine wordt gekaracteriseerd door zijn overgangsfunctie $\delta$.
    \[ \delta:\ Q \times \Gamma \rightarrow Q \times \Gamma \times \{L,R,S\} \]
    In het voorschrift van $\delta$ is staan $Q,\Gamma$ en $\{L,R,S\}$. 
    Dit zijn alledrie eindige verzamelingen. 
    Voor een gegeven grootte van $\Gamma$ en $Q$ zijn er dus maar een eindig aantal Turingmachines.
    Beginnende bij $\Gamma = \{\#,a\}$ en $|Q|=1$ kunnen we dus alle Turringmachines opschrijven in volgorde volgens hun $(|Q|\cdot|\Gamma|)$-getal.
    Dit betekent dat we de Turingmachines kunnen nummeren.
    Er bestaat dus een bijectie tussen $\mathbb{N}$ en $TM$.
  \end{proof}
\end{st}

\extra{NDTMs?}

\begin{st}
  De taal van Turing-machines is regulier.

  \begin{proof}
    Het $7$-tal uit de definitie van een Turing-machine bevat enkel eindige verzamelingen.
    De verzamelingen kunnen bovendien beschreven worden aan de hand van enkel hun notatie.
    Men kan bijgevolg een reguliere expressie opstellen voor de Turingmachine's.
  \end{proof}
\end{st}

\begin{st}
  De berekeningskracht van prolog is gelijk aan die van een turingmachine.

  \begin{proof}
    We bewijzen elke richting appart, en geven slechts een schets van een bewijs.
    \begin{itemize}
    \item Een Turingmachine kan een Prologprogramma simuleren.\\
      We tonen dit niet direct aan, maar via een omweg.
      We tonen eerst aan dat Prolog ge\"implementeerd kan worden op een Intel machine.
      SWI Prolog is een voorbeeld hiervan.
      Vervolgens tonen we aan dat een Intel machine gesimuleerd kan worden met een registermachine.
      Tenslotte tonen we aan dat een registermachine gesimuleerd kan worden met een Turingmachine.
    \item Een Prologprogramma kan een Turingmachine simuleren.\\
      Het is heel eenvoudig in te zien dat een Turingmachine gesimuleerd kan worden met een Prologprogramma.
      Er zijn veel bestaande implementaties van zo'n programma.
      Er is zelfs een \verb|turing.pl| bestand binnen SWI Prolog.\cite{prolog-tm}
    \end{itemize}
  \end{proof}
\end{st}

\section{Configuraties}
\label{sec:configuraties}

\begin{de}
  Zij $M = (Q,\Sigma,\Gamma, \delta, q_{s}, q_{a}, q_{r})$ een Turingmachine die zich in een staat $q$ bevindt tijdens de uitvoering. 
  Zij $\alpha$ de niet-$\#$ symbolen links van de schrijfeenheid en $\beta$ de niet-$\#$ symbolen onder en rechts van de schrijfeenheid.
  We noteren dan deze \term{configuratie} van $M$ als volgt.
  \[ \alpha q \beta \]
\end{de}

\begin{opm}
  Merk op dat er links en rechts van de schrijfeenheid altijd een aantal vakjes bestaan waarna alle vakjes een $\#$-symbool bevatten.
  Dit betekent echter niet dat alle symbolen tot dat niet-$\#$-symbolen zijn.
  In de bovenstaande definitie bedoelen we dus met $\alpha$/$\beta$ de symbolen voor het oneindig aantal vakjes met $\#$-symbolen begint.
\end{opm}

\begin{opm}
  De beginconfiguratie van een Turingmachine is altijd van de volgende vorm:
  \[ q_{s}\alpha \]
\end{opm}

\begin{opm}
  De eindconfiguratie van een Turingmachine is altijd van de volgende vorm:
  \[ \alpha q_{a} \beta \]
\end{opm}

\begin{opm}
  Tijdens de uitvoering van een TM zijn twee opeenvolgende configuraties verbonden door $\delta$ op \'e\'en van de volgende manieren, afhankelijk van of de turingmachine naar links gaat, naar rechts gaat, of terplaatse blijft na het schrijven.

  \begin{figure}[H]
    \centering
    \[
    \begin{array}{c|ccc}
      \delta & &\text{overgang}&\\
      \hline
      \delta(q,b) = (p,c,S) & \alpha q b \beta &\rightarrow& \alpha p a c \beta\\
      \delta(q,b) = (p,c,L) & \alpha a q b \beta &\rightarrow& \alpha p a c \beta\\ 
      \delta(q,b) = (p,c,R) & \alpha q b \beta &\rightarrow& \alpha c p \beta
    \end{array}
    \]
    \caption{Overgang van TM configuraties}
    \label{overgang-tm-configuraties}
  \end{figure}
\end{opm}

\section{Universele Turingmachine}
\label{sec:universele-turingmachine}

\begin{de}
  Een encodering van een verzameling $A$ over een alfabet $\Sigma$ is een bijectieve afbeelding $e: A \rightarrow \Sigma^{*}$ die elk element afbeeldt op een string.
\end{de}

\begin{opm}
  Idealiter introduceert een encodering geen extra informatie.
\end{opm}

\begin{de}
  Twee encoderingen $e_{1}$ en $e_{2}$ zijn equivalent als de omzetting van de ene naar de andere codering Turing-berekenbaar is.
  \[ e_{1} \sim e_{2} \Leftrightarrow e_{2} \circ e_{1}^{-1} \text{ en } e_{1} \circ e_{2}^{-1} \text{ zijn Turing-berekenbaar.}\]
\end{de}

\begin{de}
  De encodering van een object $a$ noteren we als $<a>$.
\end{de}

\begin{de}
  Een \term{universele Turingmachine} is een Turingmachine die een willekeurige ge\"encodeerde Turingmachine kan simuleren.
\end{de}

\begin{de}
  Een verzameling $V$ is \term{effectief aftelbaar} of \term{opsombaar} als en een TM bestaat die alle elementen uit $v$ (eventueel geencodeerd) \'e\'en voor \'e\'en genereert.
\end{de}

\section{De enumeratormachine}
\label{sec:de-enumeratormachine}

\begin{de}
  Een \term{enumeratormachine} is een is een $7$-tal $(Q,\Sigma,\Gamma, \delta, q_{s}, q_{a}, q_{r})$, gelijkaardig aan een Turingmachine.
  De overgangsfunctie heeft voor een enumeratormachine een andere signatuur:
  \[ \delta:\ Q \times \Gamma \rightarrow Q \times \Gamma \times \Gamma_{\epsilon}\times \{L,R,S\} \]
  Een enumeratormachine heeft bovendien een enumeratortoestand $q_{e}$, een outputband (naast de inputband) en een outputmarker.
  Wanneer de enumeratormachine een stap afrondt, schrijft hij ook nog een symbool weg naar de outputband en gaat daar ook steeds naar rechts. Wanneer de machine in toestand $q_{e}$ belandt, schrijft hij de outputmarker naar de outputband, en gaat terug over in de begintoestand $q_{s}$.
\end{de}

\begin{de}
  De taal bepaald door een enumeratormachine is de verzameling van strings die volledig op de outputband belanden, gescheiden door de outputmarker.
\end{de}

\begin{de}
  We noemen een taal \term{enumereerbaar} als er een enumeratormachine bestaat die die taal bepaalt.
\end{de}

\begin{st}
  \label{st:enumer-is-herk}
  De taal bepaald door een enumerator $E$ is herkenbaar.
  \begin{proof}
    We beschrijven een herkenner $H$ voor de taal $L$ bepaald door een enumerator $E$.
    Gegeven een string $s$ als input doet $H$ het volgende.
    $H$ laat $E$ lopen tot $E$ een string $s'$ heeft afgemaakt.
    Vervolgens kijkt $H$ of $s$ en $s'$ gelijk zijn.
    Zo ja, accepteert $H$ $s$.
  \end{proof}
\end{st}

\begin{st}
  \label{st:herk-is-enumer}
  Elke herkenbare is enumereerbaar.
  \begin{proof}
    Zij $M$ een $TM$ die een taal $L$ bepaalt.
    $L$ is dus herkenbaar.
    We construeren nu de enumerator $E$ die $L$ enumereert.
    Eerst construeren we een $TM$ $gen$ die, gegeven een input $n$, de eerste $n$ strings van $\Sigma^{*}$ op de band zet.
    Vervolgens construeren we een $TM$ $M_{n}$ die, gegeven een input $n$, de eerste $n$ stappen van $M$ uitvoert.
    Tenslotte construeren we de enumerator $E$ die de opeenvolgende natuurlijke getallen genereert, en na elk getal eerst $gen$ uitvoert, dan $M_{n}$ uitvoert, en elke string op de band zet die $M_{n}$ accepteert.
  \end{proof}
\end{st}



\end{document}
