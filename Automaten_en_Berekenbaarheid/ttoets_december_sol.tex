\documentclass[fleqn]{article}
\title{Automaten en Berekenbaarheid:\\opgave 2}
\author{Prof. B. Demoen (\url{Bart.Demoen@cs.kuleuven.be})\\ W. Van Onsem (\url{Willem.VanOnsem@cs.kuleuven.be})}
\date{12 december 2013}
\usepackage{tikz,../assignment-nl,amssymb}
\usetikzlibrary{shapes}
\newcommand{\lang}[1]{\textsc{#1}}
\newcommand{\cons}{\mbox{CONS}}
\newcommand{\NN}{\ensuremath{\mathbb{N}}}

\tikzstyle{every picture}+=[remember picture]
\everymath{\displaystyle}
\tikzstyle{na} = [baseline=-.5ex]

\begin{document}
\maketitle
\paragraph{Richtlijnen}

Je schrijft enkel op de gekleurde bladen die je gekregen hebt: op je
werkplek liggen geen andere bladen dan die gekleurde bladen en de
bladen van de cursustekst. De bladen van je cursustekst waarop
oplossingen van oefeningen of aanwijzingen tot oplossingen van
oefeningen staan, steek je ook weg. Je jas, je boekentas, je gsm
... leg je vooraan in het auditorium, zoals gebruikelijk bij een
examen. Bij het afgeven zorg je dat je naam op je bladen staat, en dat
je alles aan elkaar niet. Er zijn 3 vragen.

\begin{question}[Herkenbaar, beslisbaar, regulier, context-vrij,...?]
Waar of niet waar? Beargumenteer/bewijs.
\begin{itemize}
 \item $\mbox{EQ}_{\mbox{\small{CFG}}}=\accl{\tupl{G_1,G_2}| \mbox{Twee context-vrije grammatica's $G_1$ en $G_2$ met $\fun{L}{G_1}=\fun{L}{G_2}$}}$ is co-herkenbaar.
 \item Bepalen of er voor een gegeven Java-programma een invoer bestaat zodat de Java-emulator een gemarkeerde lijn code zal uitvoeren is beslisbaar.
 \item $\mbox{A}_{\mbox{\small{CSG}}}=\accl{\tupl{G,x}| \mbox{Een context-sensitieve grammatica $G$ en een string $s\in\Sigma^*$ met $s\in\fun{L}{G}$}}$ is herkenbaar.
 \item $\accl{\tupl{M}|M\mbox{ is een Turingmachine waarbij de taal $\fun{L}{M}$ een eindig aantal strings bevat}}$ is beslisbaar, herkenbaar, en/of co-herkenbaar.
 \item $\accl{\tupl{M}|M\mbox{ is een Turingmachine waarbij de taal $\fun{L}{M}$ minder dan $k$ strings bevat}}$ met $k$ een vaste parameter is beslisbaar, herkenbaar, en/of co-herkenbaar.
\end{itemize}
\end{question}
\begin{answer}\hfill
\begin{itemize}
 \item \textit{Waar}. Een PDA kan belissen (herkennen en co-herkennen) of een gegeven string tot de grammatica in kwestie behoort. Men kan dus een Turingmachine implementeren die in elke iteratie een string genereert en op de overeenkomstige PDA's laat lopen. Op het moment dat de PDA's een verschillend antwoord geven, reject de Turing machine.
 \item \textit{Niet waar}. Hier geldt de stelling van Rice. Een bepaalde lijn code uitvoeren is immers niet triviaal\footnote{Het is niet ``altijd'' of ``nooit'' waar, men moet minstens een analyse maken van de werking van de ingevoerde machine.} Daarnaast is de eigenschap taal-invariant. Wanneer men het Java-programma naar andere talen zou omzetten kan men de markering behouden. Eventueel verspreid over meerdere lijnen/toestanden.
 \item \textit{Waar}. Dit komt neer op het parsen van een string voor een context sensitieve grammatica. Het acceptance probleem voor context sensitieve grammatica's is beslisbaar: men ken een algoritme schrijven die voor elke string eindigt en teruggeeft of de string tot de taal behoort beschreven door de context-sensitieve grammatica.
 \item \textit{Noch herkenbaar noch co-herkenbaar}. Aan de hand van het bewijs van de stelling van Rice kunnen we zien dat de taal niet co-herkenbaar is en dus evenmin beslisbaar: $M_{\emptyset}$ voldoet niet aan de eigenschap dus moeten we de eigenschap omkeren. De taal is evenmin herkenbaar: we zouden immers $E_{TM}$ kunnen beslissen: we roepen eerst de machine $F_{TM}$ op die kijkt of de taal eindig is, wanneer deze reject, weten we dat de taal niet leeg is. Wanneer de taal wel leeg is, passen we de Turing machine aan: de machine leest eerst voorbij een string van willekeurige lengte, Bijvoorbeeld tot een komma wordt ingelezen. Indien er \'e\'en string of meer in de originele machine werd geaccepteerd, worden nu oneindig veel strings geaccepteerd, we kunnen namelijk voor elke originele string oneindig veel strings op de machine laten accepteren. We laten $F_{TM}$ op de nieuwe machine lopen. Wanneer de machine accepteert is de taal leeg, anders is de taal niet leeg. Vermits $E_{TM}$ echter niet herkenbaar is, is $F_{TM}$ dit bijgevolg evenmin.
 \item \textit{Co-herkenbaar}. Men kan immers \'e\'en of meerdere Turing machines op een Turing machine simuleren. Volgend algoritme stopt altijd wanneer de string niet tot de taal behoort: de Turing machine werkt met een oneindige lus. Wanneer de machine voor de $i$-de maal de lus uitvoert, beschouwt men voor elke string van lengte $i-1$ een simulatie op een Turingmachine. Deze simulaties hebben echter nog $0$ stappen uitgevoerd. Verder zal men de simulaties die al in vorige iteraties werden beschouwd, \'e\'en stap verderzetten. Wanneer een simulatie afloopt en accepteert, dan wordt een teller opgehoogd. Wanneer de teller gelijk wordt aan $k$ reject de Turing machine, anders blijft de machine verder uitvoeren. Het probleem is niet beslisbaar, vanwege de stelling van Rice.
\end{itemize}
\end{answer}
\begin{question}[Lambda calculus]\brak{f\ x}
\begin{itemize}
 \item Ontwerp een zuivere $\lambda$-expressie hier $\mbox{il}$ genoemd zodat voor elke $i\geq 0$ geldt: $\fun{\mbox{head}}{\fun{\mbox{tail}^i}{\mbox{il}}}=\fun{f^i}{x}$. Mocht men dus de expressie $\mbox{il}$ evalueren zou dit resulteren in volgende oneindige expressie:
 \begin{equation}
  \fun{\cons}{x,\fun{\cons}{ \fun{f}{x}, \fun{\cons}{ \fun{f}{\fun{f}{x}}, \fun{\cons}{ \fun{f}{\fun{f}{\fun{f}{x}}} , \ldots } } }}
 \end{equation}
 Evalueer ook de expressie \funm{head}{\funm{tail}{\mbox{il}}} volgens reductie in normaalorde.
 \item In lambda-calculus zijn variabelen niet altijd gebonden. Omcirkel bij onderstaande expressie de gebonden variabelen en wijs met pijl de overeenkomstige lambda aan. Bijvoorbeeld:
 \[
 \tikz[baseline]{\node(l){$\lambda$};}\ \tikz[baseline]{\node(y){x};}\ \tikz[baseline]{\node(x){.};}\ \tikz[baseline]{\node[draw,circle](x){x};}
 \]
\begin{tikzpicture}[overlay]
\path[->] (x) edge [bend left] (l);
\end{tikzpicture}
 Ongebonden variabelen duid je aan met een hoedje ($\hat{x}$).
 \begin{equation}
  +\ y\ \lambdacal{x}{+\ y\ \lambdacal{y}{\lambdacal{x}{+\ \brak{+\ y\ y}\ x}\ y}}\lambdacal{y}{+\ y\ x}
 \end{equation}
\end{itemize}
\end{question}
\begin{answer}\hfill
\begin{itemize}
 \item De expressie voor $\mbox{il}$ is de volgende:
\begin{equation}
Y\ \ \lambdacal{g\ f\ x}{\cons\ x \brak{g\ f\ \brak{f\ x}}}
\end{equation} ofwel voluit
\begin{equation}
\lambdacal{g}{\lambdacal{x}{g\ x\ x}\lambdacal{x}{g\ x\ x}}\ \lambdacal{g\ f\ x}{\lambdacal{a\ b\ f}{f\ a\ b}\ x \brak{g\ f\ \brak{f\ x}}}
\end{equation}
We zullen de evaluatie met behulp van de compacte notatie uitvoeren en enkel operatoren ontbinden wanneer dit nodig is:
\begin{eqnarray}
\mbox{head}\ \brak{\mbox{tail}\ \brak{\mbox{li}\ f\ x}}\hfill\\
\lambdacal{c}{c\ \lambdacal{a\ b}{a}}\ \brak{\mbox{tail}\ \brak{ \mbox{li}\ f\ x }}\hfill\\
\brak{\mbox{tail}\ \brak{ \mbox{li}\ f\ x }}\ \lambdacal{a\ b}{a}\hfill\\
\brak{\lambdacal{c}{c\ \lambdacal{a\ b}{b}}\ \brak{ \mbox{li}\ f\ x }}\ \lambdacal{a\ b}{a}\hfill\\
\brak{\brak{ \mbox{li}\ f\ x } \lambdacal{a\ b}{b}}\ \lambdacal{a\ b}{a}\hfill\\
\brak{\brak{ \brak{ Y\ \lambdacal{g\ f\ x}{\cons\ x \brak{g\ f\ \brak{f\ x}}} }\ f\ x } \lambdacal{a\ b}{b}}\ \lambdacal{a\ b}{a}\hfill\\
\brak{\brak{ \lambdacal{f\ x}{\cons\ x \brak{\mbox{li}\ f\ \brak{f\ x}}}\ f\ x } \lambdacal{a\ b}{b}}\ \lambdacal{a\ b}{a}\hfill\\
\brak{\cons\ x \brak{\mbox{li}\ f\ \brak{f\ x}} \lambdacal{a\ b}{b}}\ \lambdacal{a\ b}{a}\hfill\\
\brak{\lambdacal{h}{h\ x \brak{\mbox{li}\ f\ \brak{f\ x}}} \lambdacal{a\ b}{b}}\ \lambdacal{a\ b}{a}\hfill\\
\brak{\lambdacal{a\ b}{b}\ x \brak{\mbox{li}\ f\ \brak{f\ x}} }\ \lambdacal{a\ b}{a}\hfill\\
\brak{\brak{ Y\ \lambdacal{g\ f\ x}{\cons\ x \brak{g\ f\ \brak{f\ x}}} }\ f\ \brak{f\ x} }\lambdacal{a\ b}{a}\hfill\\
\brak{\lambdacal{f\ x}{\cons\ x \brak{\mbox{li}\ f\ \brak{f\ x}}}\ f\ \brak{f\ x} }\lambdacal{a\ b}{a}\hfill\\
\cons\ \brak{f\ x} \brak{\mbox{li}\ f\ \brak{f\ \brak{f\ x}}}\lambdacal{a\ b}{a}\hfill\\
\lambdacal{h}{h\ \brak{f\ x} \brak{\mbox{li}\ f\ \brak{f\ \brak{f\ x}}}}\lambdacal{a\ b}{a}\hfill\\
\lambdacal{a\ b}{a}\ \brak{f\ x} \brak{\mbox{li}\ f\ \brak{f\ \brak{f\ x}}}\hfill\\
f\ x\hfill
\end{eqnarray}
\item Zie onderstaande expressie: \[
\tikz[baseline]{\node(p1){+};}\ \enctikz{temp}{\hat{y}}\ \lambdacaltikz{x1}{x}{\enctikz{tmp}{\enctikz{tmp}{+\ \hat{y}}}\ \lambdacaltikz{y1}{y}{\lambdacaltikz{x2}{x}{\enctikz{tmp}{+}\ \enctikz{tmp}{(+}\ \enctikzd{y1l1}{y}\ \enctikzd{y1l2}{y}\enctikz{tmp}{)}\ \enctikzd{x1l1}{x}}\ \enctikzd{y1l3}{y}}}\lambdacaltikz{y2}{y}{\enctikz{tmp}{+}\ \enctikzd{y2l1}{y}\ \enctikz{tmp}{\hat{x}}}
 \]
\begin{tikzpicture}[overlay]
\foreach \x/\y/\m in {y1l1/y1/left,y1l2/y1/right,x1l1/x1/left,y1l3/y1/right,y2l1/y2/left} {
  \path[->] (\x) edge [bend \m] (la\y);
}
\end{tikzpicture}
\end{itemize}
\end{answer}
\begin{question}[Primitief recursief en \textsc{for}-programma's]
Primitief recursieve programma's zijn even sterk als \textsc{for}-programma's: programma's die als enig controle-mechanisme een \verb+for+-lus kunnen defini\"eren. De grenzen (bijvoorbeeld \verb+m+ en \verb+n+) moet voor het binnengaan van de for-lus gekend zijn en zowel de teller als de grens mogen niet aangepast worden tijdens uitvoer van de lus. Recursie is uiteraard ook verboden (anders zou men via recursie for-lussen met aanpasbare teller/grenzen kunnen emuleren). Een geldig \textsc{for}-programma is bijvoorbeeld:
\begin{verbatim}
function f(m,n)
    for i = m:n
        m = m*n;
    endfor
    for i = 0:m
        n = n+i;
    endfor
    return n;
endfunction
\end{verbatim}
Een ongeldig \textsc{for}-programma is bijvoorbeeld:
\begin{verbatim}
function f(m,n)
    for i = m:n
        i = i*m;
    endfor
    for i = 0:m
        m = m-i;
        i = 2*i;
        n = f(m+1,n-1);
    endfor
    return n;
endfunction
\end{verbatim}
Beschrijf aan de hand van een reeks transformatieregels hoe men een gegeven primitief-recursieve functie kan omzetten naar een \textsc{for}-programma.
\end{question}
\begin{answer}
De basisfuncties zoals $\mbox{nul}$, $\mbox{suc}$ en $\mbox{p}_i^n$ hebben een straight-forward implementatie:
\begin{verbatim}
function nul(n)
    return 0;
endfunction

function succ(n)
    return n+1;
endfunction
\end{verbatim}
De $i$-de projectie uit $n$ elementen:
\begin{verbatim}
function pin(x1,x2,...,xn)
    return xi;
endfunction
\end{verbatim}
De compositie $\mbox{Cn}\left[f,g_1,g_2,\ldots,g_m\right]$ met $f:\NN^m\rightarrow\NN$ en voor elke $g_i$, $g_i:\NN^k\rightarrow\NN$ wordt dan als volgt gedefinieerd:
\begin{verbatim}
function cn(x1,x2,...,xk)
    y1 = g1(x1,x2,...,xk);
    y2 = g2(x1,x2,...,xk);
    //...
    ym = gm(x1,x2,...,xk);
    return f(y1,y2,...,ym);
endfunction
\end{verbatim}
Tot slot de primitieve recursie functie $\mbox{Pr}\left[f,g\right]$ met $f:\NN^k\rightarrow\NN$ en $g:\NN^{k+2}\rightarrow\NN$:
\begin{verbatim}
function pr(x1,x2,...,xk,l)
    y = f(x1,x2,...,xk);
    for i=1:l
        y = g(x1,x2,...,xk,i,y);
    endfor
    return y;
endfunction
\end{verbatim}
\end{answer}
Indien er in een programma meerdere primitief recursieve en/of compositie-functies voorkomen, moet men natuurlijk verschillende namen defini\"eren.
\paragraph{Voorbeeld}
Bij wijze van voorbeeld: de $\mbox{som}$-functie is gedefinieerd als: $\mbox{Pr}\left[p^1_1,\mbox{Cn}\left[\mbox{succ},p^3_3\right]\right]$. We bekomen dan volgende code:
\begin{verbatim}
function succ(n)
    return n+1;
endfunction

function p11(x1)
    return x1;
endfunction

function p33(x1,x2,x3)
    return x3;
endfunction

function cn(x1,x2,x3)
    y1 = p33(x1,x2,x3);
    return succ(y1);
endfunction

function pr(x1,x2)
    y = p11(x1);
    for i = 1:x2
      y = cn(x1,i,y);
    endfor
    return y;
endfunction

\end{verbatim}
\end{document}