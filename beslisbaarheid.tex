\documentclass[main.tex]{subfiles}
\begin{document}

\chapter{Beslisbaarheid}
\label{cha:beslisbaarheid}

\section{De stelling van Rice}
\label{sec:de-stelling-van}

\begin{de}
  Een \term{eigenschap} $P$ van een $\phi$is een predicaat dat de verzameling van $\phi$s in twee verdeelt: $Pos_{P}$, de verzameling van $\phi$ die $P$ bezitten en $Neg_{P}$, de verzameling van $\phi$ die $P$ niet bezitten.
  \[ Pos_{P} = \{ \phi \ |\ P(\phi) \} \quad\text{ en }\quad Neg_{P} = \{ \phi \ |\ \neg P(\phi) \} \]
\end{de}

\begin{de}
  Een \term{niet-triviale eigenschap} is een eigenschap waarvoor zowel $Pos_{P}$ als $Neg_{P}$ niet leeg is.
\end{de}

\begin{de}
  Een \term{taal-invariante eigenschap} van een machine $phi$ is een eigenschap die enkel afhankelijk is van de taal die $\phi$ bepaalt.
  \[ \forall M_{1},M_{2}:\ L_{M_{1}} = L_{M_{2}} \Rightarrow (P(M_{1}) \Leftrightarrow P(M_{2})) \]
\end{de}

\begin{st}
  \label{st:eerste-stelling-van-rice}
  Eerste \term{stelling van Rice}\\
  Voor elke niet-triviale, taal-invariante eigenschap $P$ van Turingmachines is noch $Pos_{P}$ noch $Neg_{P}$ beslisbaar.

  \begin{proof}
    Bewijs uit het ongerijmde\\
    Stel dat er een niet-triviale taal-invariante eigenschap $P$ bestaat en een beslisser $B$ voor $Pos_{P}$ (en dus ook een beslisser voor $Neg_{P}$).
    We construeren vanuit $B$ een beslisser $A$ voor $A_{TM}$.
    Stel dat $M_{\phi}$ (de machine die de lege taal beslist) de eigenschap $P$ niet heeft.
    (Indien dat diet zo is, verander $P$ in zijn negatie. Dan is het immers wel zo.)
    Omdat $P$ niet-triviaal is bestaat er een taal $L_{X}$ zodat $X$ een Turingmachine is met de eigenschap $P$ die $L_{X}$ bepaalt.
    $A$ krijgt als invoer een string van de vorm $<M,s>$ en doet het volgende:
    \begin{itemize}
    \item $A$ construeert een hulpmachine $H_{M,s}$ die het volgende doet bij invoer $w$:
      \begin{itemize}
      \item $H$ laat $M$ lopen op $s$.
      \item Als $M$ $s$ accepteert, dan laat $H$ $X$ lopen op $w$ en geeft het resultaat terug.
        In het andere geval verwerpt $H$ $s$.
      \end{itemize}
    \item $A$ laat nu $B$ los op $H_{M,s}$. en geeft het resultaat terug.
    \end{itemize}
    Merk eerst op dat $H_{M,s}$ ofwel de lege taal accepteert, ofwel $L_{X}$.
    Als $w$ een string is die niet door $M$ geaccepteerd wordt, dan zal $H$ die ook niet accepteren.
    Als $w$ door $m$ geaccepteerd wordt, maar niet in $X$ zit, dan zal $H$ die ook niet accepteren.
    Als $w$ door $m$ geaccepteerd wordt, en ook in $X$ zit, dan zal $H$ $w$ accepteren.
    Of $H_{M,s}$ $L_{X}$ accepteert of niet hangt af van of $M$ al dan niet $s$ accepteert.
    We bekijken nu wat het inhoudt als $A$ als invoer $<M,s>$ krijgt:
    \[
    \begin{array}{rl}
      A \text{ accepteert } <M,s> &\Leftrightarrow B \text{ accepteert } H_{M,s}\\
      &\Leftrightarrow H_{M,s} \text{ heeft eigenschap } P\\ 
      &\Leftrightarrow H_{M,s} \text{ accepteert } L_{X}\\
      &\Leftrightarrow M \text{ accepteert } s
    \end{array}
    \]
    Dit betekent precies dat $A$ een beslisser is voor $A_{TM}$.
    Omdat $A$ niet beslisbaar is kan $B$ niet bestaan en is $Pos_{P}$ dus ook niet beslisbaar.
  \end{proof}
\end{st}

\extra{tweede stelling van rice}
\extra{voor welke vorige stellingen kunnen we de stelling van rice gebruiken?}

\section{Reductie}
\label{sec:reductie}

\begin{de}
  Een functie $f$ is \term{Turing-berekenbaar} als er een Turingmachine bestaat die voor elke $s$ in het domein van $f$, $f(s)$ op de band zet.
\end{de}

\begin{de}
  Omwille van du Church-Turing these korten we ``Turing-berekenbaar'' vaak af tot ``berekenbaar''.
\end{de}

\begin{de}
  We zeggen dat een taal $L_1$ (over $\Sigma_1$) gereduceert kan worden naar $L_2$ (over $\Sigma_2$) als er een afbeelding $f$ als volgt:
  \[  f:\ \Sigma_1\rightarrow \Sigma_1  \]
  \begin{itemize}
    \item $f(L_1) \subseteq L_2$
    \item $f(\overline{L_1}) \subseteq \overline{L_2}$
    \item $f$ is Turing-berekenbaar
  \end{itemize}
  \[ L_1\le_m L_2 \]
\end{de}

\begin{st}
  Zij $L_1$ en $L_2$ talen over, respectievelijk, $\Sigma_1$ en $\Sigma_2$.
  \[ L_1\le_m L_2 \Rightarrow (L_2 \text{ is beslisbaar } \Rightarrow L_1 \text{ is beslisbaar }) \]

  \begin{proof}
    Zij $L_{1}$ reduceerbaar naar $L_{2}$ en $L_{2}$ beslisbaar.
    Gegeven een beslisser $B_{2}$ voor $L_{2}$ zullen we een beslisser $B_{1}$ voor $L_{1}$ construeren.
    $B_{1}$ zal als invoer een string $s$ verwachten en moeten beslissen of $s$ tot $L_{1}$ behoort.
    \begin{itemize}
    \item $B_{1}$ berekent de afbeelding $f(s)$ van $s$ onder $f$.
      Dit kan want uit de definitie van reduceerbaarheid volgt dat $f$ Turing-berekenbaar is.
    \item $B_{1}$ gaat (in eindige tijd) na of $f(s)$ tot $L_{2}$ behoort door $B_{2}$ op te roepen en geeft het resultaat terug.
    \end{itemize}
    Precies omdat $L_{1}$ op een deel van $L_{2}$ wordt afgebeeld en $\overline{L_{1}}$ op een deel van $\overline{L_{2}}$ komt het resultaat van $B_{2}$ overeen met het resultaat van $B_{1}$.
  \end{proof}
\end{st}


\begin{st}
  Zij $L_1$ en $L_2$ talen over, respectievelijk, $\Sigma_1$ en $\Sigma_2$.
  \[ L_1\le_m L_2 \Rightarrow (L_2 \text{ is herkenbaar } \Rightarrow L_1 \text{ is herkenbaar }) \]

  \begin{proof}
    Zij $L_{1}$ reduceerbaar naar $L_{2}$ en $L_{2}$ herkenbaar.
    Gegeven een herkenner $H_{2}$ voor $L_{2}$ zullen we een herkenner $H_{1}$ voor $L_{1}$ construeren.
    $H_{1}$ zal als invoer een string $s$ verwachten en moeten herkennen of $s$ tot $L_{1}$ behoort.
    \begin{itemize}
    \item $H_{1}$ berekent de afbeelding $f(s)$ van $s$ onder $f$.
      Dit kan want uit de definitie van reduceerbaarheid volgt dat $f$ Turing-berekenbaar is.
    \item $H_{1}$ gaat na of $f(s)$ tot $L_{2}$ behoort door $H_{2}$ op te roepen en geeft het resultaat terug.
    \end{itemize}
    Precies omdat $L_{1}$ op een deel van $L_{2}$ wordt afgebeeld komt het resultaat van $B_{2}$ overeen met het resultaat van $B_{1}$.
  \end{proof}
\end{st}

\begin{opm}
  In het geval dat een taal $L_{1}$ reduceerbaar is naar een taal $L_{2}$ en $L_{2}$ herkenbaar, loopt de bovenstaande constructie ook oneindig door voor alle strings $f(s)$ waarvoor de herkenner voor $L_{2}$ oneindig doorloopt.
\end{opm}

\begin{gev}
  \label{gev:reductie-niet-beslisbaarheid}
  Zij $L_{1}$ een taal die reduceerbaar is naar een taal $L_{2}$ met $L_{1}$ niet herkenbaar/niet beslisbaar, dan is $L_{2}$ ook niet herkenbaar/beslisbaar.
  \zb
\end{gev}

\begin{st}
  \label{st:eq-tm-niet-besl}
  $EQ_{TM}$ is niet beslisbaar.

  \begin{proof}
    We weten dat $E_{TM}$ niet beslisbaar is.
    We reduceren $E_{TM}$ naar $EQ_{TM}$ met de functie $f$:
    \[
    <M> \mapsto <M,M_{\phi}>
    \]
    $f$ is duidelijk Turing-berekenbaar. $M_{\phi}$ is immers eindig.
    Omdat $E_{TM}$ niet beslisbaar is, is $EQ_{TM}$ ook niet beslisbaar.\gevref{gev:reductie-niet-beslisbaarheid}
  \end{proof}
\end{st}

\begin{st}
  \label{st:reduceerbaarheid-taal-complement}
  Zij $A$ en $B$ twee talen.
  \[ A \le_m B \Rightarrow \overline{A} \le_m \overline{B} \]

  \begin{proof}
    Zij $A \le_{m} B$, dan bestaat er een afbeelding $f$ die $L_{1}$ afbeeldt op een deel van $L_{2}$ en $\overline{L_{1}}$ op een deel van $\overline{L_{2}}$.
    Diezelfde afbeelding houdt in dat $\overline{A}$ reduceerbaar is naar $\overline{B}$.
    Het complement van het complement van een verzameling is immers opnieuw de originele verzameling.
  \end{proof}
\end{st}


\section{Orakelmachines}
\label{sec:orakelmachines}

\begin{de}
  Een orakel voor een taal $L$ is een machine die in een eindige hoeveelheid tijd kan beslissen of een string tot $L$ behoort.
\end{de}

\begin{de}
  Een orakelmachine $O^{L}$ is een machine die als subroutine een orakel voor de taal $L$ kan aanroepen.
\end{de}

\begin{st}
  Er bestaat een orakelmachine $O^{A_{TM}}$ die $E_{TM}$ beslist.

  \begin{proof}
    Bij invoer $<M>$ gaat $O^{A_{TM}}$ als volgt te werk.
    \begin{itemize}
    \item Constreert een Turingmachine $P$ die $M$ laat lopen op elke string en enkel zijn invoer aanvaardt als $M$ minstens \'e\'en string aanvaardt.
    \item $O^{A_{TM}}$ vraagt aan het orakel voor $A_{TM}$ of $<P,x>$ tot $A_{TM}$ behoort (voor een willekeurige $x$) en geeft het tegengestelde terug.
    \end{itemize}
    $O^{A_{TM}}$ beslist op deze manier of een machine $M$ al dan niet de lege taal bepaalt.
  \end{proof}
\end{st}

\begin{de}
  Een taal $A$ is \term{Turingreduceerbaar} naar een taal $B$ als $A$ beslisbaar is ten opzichte van $B$.
  Dit betekent dat er een orakelmachine $O^{B}$ bestaat die $A$ beslist.
  \[ A \le_{T} B \]
\end{de}

\begin{st}
  Als $A$ Turingreduceerbaar is ten opzichte van $B$ en $B$ beslisbaar, dan is $A$ beslisbaar.

\extra{bewijs}
\end{st}

\begin{st}
  Als $A$ reduceerbaar is naar $B$, dan is $A$ ook Turingreduceerbaar naar $B$.
\end{st}

\begin{gev}
  $\le_{m}$ is dus fijner dan $\le_{T}$.
\extra{bewijs}
\end{gev}

\section{... in verband met reguliere talen}
\label{sec:verb-met-regul}

\begin{de}
  \label{de:a-dfa}
  $A_{DFA}$ is de taal van koppels $(D,s)$ waarbij $D$ een DFA is die $s$ aanvaardt.
  \[ A_{DFA} = \{ <D,s> \ |\ D \text{ is een DFA en } s \in L_{D} \} \]
\end{de}

\begin{st}
  \label{st:a-dfa-besl}
  $A_{DFA}$ is beslisbaar.

  \begin{proof}
    Een beslisser $B$ voor $A_{DFA}$ krijgt als input $<D,s>$ met $D$ een $DFA$.
    De beslisser simuleert $D$, en accepteert $s$ als en slechts als $D$ $s$ accepteert.
  \end{proof}
\end{st}

\begin{de}
  \label{de:a-re}
  $A_{RE}$ is de taal van koppels $(R,s)$ waarbij $R$ een reguliere expressie is die $s$ genereert.
  \[ A_{RE} = \{ <R,s> \ |\ R \text{ is een reguliere expressie en } s \in L_{R} \} \]
\end{de}

\begin{st}
  \label{st:a-re-besl}
  $A_{RE}$ is beslisbaar.

  \begin{proof}
    Een beslisser $B$ voor $A_{RE}$ krijgt als input een reguliere expressie $R$ en een string $s$ als $<R,s>$.
    De beslisser construeert dan een DFA voor $R$\gevref{gev:reguliere-taal-DFA} en roept daarna een beslisser voor $A_{DFA}$ op.\stref{st:a-dfa-besl}
  \end{proof}
\end{st}

\begin{de}
  \label{de:a-nfa}
  $A_{NFA}$ is de taal van koppels $(N,s)$ waarbij $N$ een NFA is die $s$ aanvaardt.
  \[ A_{NFA} = \{ <N,s> \ |\ N \text{ is een NFA en } s \in L_{N} \} \]
\end{de}

\begin{st}
  \label{st-a-nfa-besl}
  $A_{NFA}$ is beslisbaar.

  \begin{proof}
    Een beslisser $B$ voor $A_{NFA}$ krijgt als input $<N,s>$.
    $B$ zet $N$ eerst om naar een DFA\stref{st:nfa-naar-dfa} en roept daarna een beslisser voor $A_{DFA}$ op.\stref{st:a-dfa-besl}
  \end{proof}
\end{st}

\begin{de}
  \label{de:e-dfa}
  $E_{DFA}$ is de taal van alle machines $D$ die de lege taal bepalen.
  \[ E_{DFA} = \{ <D,s>\ |\ D \text{ is een DFA en } L_{D} = \emptyset \} \]
\end{de}

\begin{st}
  \label{st:e-dfa-besl}
  $E_{DFA}$ is beslisbaar.

  \begin{proof}
    We construeren een beslisser $B$ voor $E_{DFA}$.
    De beslisser krijg (de encodering van) een DFA $D$ als invoer.
    De beslisser vormt $D$ om tot een minimale DFA $D'$.\alref{al:dfa-minimalisatie}
    Als $D$ (en dus ook $D'$) de lege taal bepaalt, is $D'$ isomorf met de $NFA_{\phi}$ \figref{fig:nfa_phi}.
    Dat is eenvoudig beslisbaar. $B$ moet immers maar een eindig aantal mogelijke bijecties afgaan.
  \end{proof}
\end{st}

\begin{de}
  \label{de:eq-dfa}
  $EQ_{DFA}$ is de taal van alle koppels DFA's die dezelfde taal bepalen.
  \[ EQ_{DFA} = \{ <D_{1},D_{2}> \ |\ D_{1},D_{2} \text{ zijn DFA's en } L_{D_{1}} = L_{D_{2}} \} \]  
\end{de}

\begin{st}
  \label{st:eq-dfa-besl}
  $EQ_{DFA}$ is beslisbaar.

  \begin{proof}
    We construeren een beslisser $B$ voor $EQ_{DFA}$ als volgt:
    $B$ krijgt als input $<D_{1},D_{2}>$ en construeert vervolgens de DFA $D$ die het symmetrisch verschil bepaalt van $L_{D_{1}}$ en $L_{D_{2}}$.\deref{de:symmetrisch-verschil-dfas}
    Vervolgens beslist $B$ of $D$ de lege taal bepaalt.\stref{st:e-dfa-besl}
  \end{proof}
\end{st}

\extra{$H_{DFA}$}
\extra{$E_{RE}$}
\extra{$EQ_{RE}$}
\extra{$ES_{RE}$}
\extra{$REGULAR_{RE}$}
\extra{$ALL_{RE}$}
\extra{$FINITE_{RE}$}

\section{... in verband met contextvrije talen}
\label{sec:verb-met-cont}

\begin{de}
  \label{de:a-cfg}
  $A_{CFG}$ is de taal van koppels $(C,s)$ waarbij $C$ een contextvrije grammatica is die $s$ genereert.
  \[ A_{CFG} = \{ <G,s> \ |\ G \text{ is een CFG en } s \in L_{C} \} \]
\end{de}

\begin{st}
  \label{st:a-cfg-besl}
  $A_{CFG}$ is beslisbaar.

  \begin{proof}
    We construeren een beslisser $B$ voor $A_{CFG}$.
    $B$ krijgt als input een string $<G,s>$ en converteert eerst $G$ naar Chomsky Normaal Vorm.
    Vervolgens genereert $G$ alle mogelijke strings afleidingslengte $2|s| - 1$.
    Dat zijn er eindig veel.
    Als $s$ erbij zit, genereert $G$ $s$.\stref{st:chomsky-normaalvorm-afleidingslengte}
    $B$ accepteert $<G,s>$ dan, anders niet.
  \end{proof}
\end{st}

\begin{de}
  \label{de:e-cfg}
  $E_{CFG}$ is de taal van CFG's $C$ die de lege taal bepalen.
  \[ E_{CFG} = \{ <C,s>\ |\ D \text{ is een CFG en } L_{C} = \emptyset \} \]
\end{de}

\begin{st}
  \label{st:e-cfg-besl}
  $E_{CFG}$ is beslisbaar.

  \begin{proof}
    We construeren een beslisser $B$ voor $E_{CFG}$.
    $B$ krijgt als input een string $<G,s>$ en gaat als volgt te werk:
    $B$ verwijdert algoritmisch regels uit $G$ zodat de resulterende CFG equivalent is, maar $B$ makkelijker een beslissing laat maken.
    Voor alle regels $A \rightarrow \alpha$ waarbij $\alpha$ enkel bestaat uit eindsymbolen verwijdert $B$ alle regels waarin $A$ aan de linkerkant voorkomt en vervangt $B$ $A$ door $\alpha$ in alle regels waar $A$ aan de rechterkant voorkomt.
    Als het startsymbool verwijderd wordt verwerpt $B$ $<G,s>$, want dan kon uit $S$ een string afgeleid worden.
    Wanneer er geen regels meer zijn van de beschreven vorm accepteert $B$ $<G,s>$.
  \end{proof}
\end{st}

\begin{de}
  \label{de:es-cfg}
  $ES_{CFG}$ is de taal van alle CFG's die de lege string aanvaarden.
  \[ ES_{CFG} = \{ <C> \ |\ C \text{ is een CFG en } \epsilon \in L_{C}\} \]
\end{de}

\begin{st}
  \label{st:es-cfg-besl}
  $ES_{CFG}$ is beslisbaar.

  \begin{proof}
    We construeren een beslisser $B$ voor $ES_{CFG}$.
    $B$ krijgt als input een string $<G,s>$ en gaat als volgt te werk:
    $B$ transformeert $G$ naar zijn Chomsky Normaal Vorm en kijkt na of er een regel van de vorm $S \rightarrow \epsilon$ overblijft.
    Zo ja accepteert $B$, anders verwerpt $B$.
  \end{proof}

  \begin{proof}
    Een beslisser voor $ES_{CFG}$ roept een beslisser voor $A_{CFG}$ op met als invoer de gegeven CFG en de lege string.\stref{st:a-cfg-besl}
  \end{proof}
\end{st}

\begin{st}
  \label{st:contextvrije-taal-besl}
  Een contextvrije taal $L$ is beslisbaar.

\extra{bewijs}
\end{st}

\begin{st}
  \label{st:eq-cfg-niet-herk}
  $EQ_{CFG}$ is niet herkenbaar.
\TODO{bewijs intuitief p 103}
\end{st}

\begin{st}
  \label{st:eq-cfg-coherk}
  $EQ_{CFG}$ is co-herkenbaar.
\extra{bewijs}
\end{st}

\begin{gev}
  \label{gev:eq-cfg-niet-besl}
  $EQ_{CFG}$ is niet beslisbaar.

  \begin{proof}
    $EQ_{CFG}$ is niet herkenbaar, dus zeker niet beslisbaar.
  \end{proof}
\end{gev}

\begin{st}
  \label{st:all-cfg-niet-herk}
  $ALL_{CFG}$ is niet herkenbaar. \zb
\end{st}

\begin{st}
  \label{st:all-cfg-coherk}
  $ALL_{CFG}$ is co-herkenbaar.
\extra{bewijs}
\end{st}

\begin{gev}
  \label{gev:all-cfg-niet-besl}
  $ALL_{CFG}$ is niet beslisbaar.

  \begin{proof}
    $ALL_{CFG}$ is niet herkenbaar, dus zeker niet beslisbaar.
  \end{proof}
\end{gev}

\extra{$H_{CFG}$}
\extra{$REGULAR_{CFG}$}
\extra{$FINITE_{CFG}$}

\section{... in verband met contextsensitieve talen}
\label{sec:-verb-met-csls}


\begin{de}
  \label{de:linear-begrensde-automaat}
  Een \term{lineair begrensde automaat} (\term{LBA}) is een Turingmachine die niet leest of schrijft buiten een deel van de geheugenstrook dat de initiele invoer bevat.
\end{de}

\begin{st}
  In een equivalente definitie leest of schrijft een LBA niet buiten een deel van de band dat een constante factor groter is dan de invoer.
\zb
\end{st}

\begin{st}
  De taal die alle Turingmachines bevat die ook een LBA zijn is niet beslisbaar.
\extra{bewijs}
\end{st}

\begin{de}
  \label{de:a-lba}
  $A_{LBA}$ is de taal van alle koppels $(L,s)$ waarbij $L$ een LBA is die $s$ aanvaardt.
  \[ A_{LBA} = \{ <L,s> \ |\ L \text{ is een LBA en } s \in L_{L}\} \]
\end{de}

\begin{st}
  \label{st:a-lba-is-besl}
  $A_{LBA}$ is beslisbaar.

  \begin{proof}
    We construeren een beslisser $B$ voor $A_{LBA}$.
    $B$ verwacht een string van de vorm $<L,s>$.
    We definieren eerst een paar getallen:
    \begin{itemize}
    \item $n$: de lengte van de invoerstring.
    \item $q$: het aantal toestanden van $L$.
    \item $b$: het aantal elementen van het bandalfabet van $L$.
    \end{itemize}
    Merk nu op dat het aantal strings dat mogelijk op de band kan staan tijdens de uitvoering van $L$ kleiner of gelijk is aan $b^{n}$.
    De leeseenheid kan bovendien hoogstens bij elk van de symbolen staan en de machine kan zich hoogstens in elk van zijn toestanden bevinden tijdens zijn uitvoering.
    Het aantal mogelijke configuraties van $L$ tijdens zijn uitvoering is dus kleiner of gelijk aan $qnb^{n}$ (eindig!).
    De beslisser $B$ voor $A_{LBA}$ gaat als volgt te work:
    \begin{itemize}
    \item $B$ berekent $m = qnb^{n}$.
    \item $B$ laat $M$ lopen op $s$ voor maximaal $m$ stappen.
    \item $B$ accepteert $s$ als en slechts als $M$ de string $s$ accepteerde binnen de $m$ stappen.
    \end{itemize}
    Als $M$ de string $s$ nog niet geaccepteerd had binnen de $m$ stappen betekent dat dat $M$ in een lus terecht was gekomen en dan zou $M$ $s$ ook niet accepteren.
  \end{proof}
\end{st}

\begin{de}
  \label{de:e-lba}
  $E_{LBA}$ is de taal van alle LBA's die geen enkele string aanvaarden.
  \[ E_{LBA} = \{ <L> \ |\ L \text{ is een LBA en } L_{L} = \emptyset \} \]
\end{de}

\begin{st}
  \label{st:e-lba-niet-herk}
  $E_{LBA}$ is niet herkenbaar.
\TODO{bewijs p 107}
\end{st}

\begin{st}
  \label{st:e-lba-coherk}
  $E_{LBA}$ is co-herkenbaar.
\extra{bewijs}
\end{st}

\begin{gev}
  $E_{LBA}$ is beslisbaar.
\TODO{bewijs}
\end{gev}

\extra{p 108 acceptance, halting en leegheid van LBA,PDA en TM}

\extra{$A_{CSG}$}
\extra{$H_{CSG}$}
\extra{$E_{CSG}$}
\extra{$EQ_{CSG}$}
\extra{$ES_{CSG}$}
\extra{$REGULAR_{CSG}$}
\extra{$ALL_{CSG}$}
\extra{$FINITE_{CSG}$}

\section{...  in verband met Turing machines}
\label{sec:verb-met-tms}

\subsection{Het acceptatie probleem}
\label{sec:het-acceptatie-probleem}

\begin{de}
  \label{de:a-tm}
  $A_{TM}$ is de taal van koppels $(M,s)$ waarbij $M$ een Turingmachine is die $s$ aanvaardt.
  \[ A_{TM} = \{ <M,s> \ |\ M \text{ is een Turingmachine en } s \in L_{M} \} \]
\end{de}

\begin{st}
  \label{st:a-tm-herk}
  $A_{TM}$ is herkenbaar.
  \begin{proof}
    De herkenner $H$ voor $A_{TM}$ laat bij input $<M,s>$ $M$ lopen op $s$.
    Als $M$ $s$ accepteert, dan accepteert $H$ $<M,s>$.
  \end{proof}
\end{st}

\begin{st}
  \label{st:a-tm-niet-besl}
  $A_{TM}$ is niet beslisbaar.

  \begin{proof}
    Bewijs uit het ongerijmde.\\
    Stel dat er een beslisser $B$ bestaat voor $A_{TM}$.
    Dit betekent dat de TM $B$ de string $<M,s>$ accepteert als $M$ bij input $s$ zou stoppen in een accepterende toestand.
    $B$ zou bovendien de string $<M,s>$ verwerpen als $M$ bij input $s$ zou stoppen in een onaanvaardbare toestand, of niet zou stoppen.
    We construeren nu een machine (TM) $C$ die voor een contradictie moet zorgen.
    $C$ accepteert een string $<M>$ (met $M$ een $TM$) als $B$ $<M,<M>>$ niet accepteert.
    $C$ verwerpt een string $<M>$ als $B$ $<M,<M>>$ verwerpt.
    We beschouwen nu wat $C$ doet met de string $<C>$.
    Stel dat $C$ $<C>$ zou accepteren, dan betekent dat dat $B$ $<C<C>>$ accepteert (volgens de definitie van $B$), want $C$ is een machine die de string $<C>$ accepteert.
    Het betekent echter dat $C$ $<C>$ moet verwerpen(, volgens de definitie van $C$).
    \[ C \text{ accepteert } <C> \quad\Rightarrow\quad B \text{ accepteert } <C<C>> \quad\Rightarrow\quad C \text{ verwerpt } <C> \]
    Stel omgekeerd dat $C$ $<C>$ zou verwerpen, dan betekent dat dat $B$ $<C<C>>$ verwerpt (volgens de definitie van $B$), want $C$ is een machine die de string $<C>$ accepteert.
    Dat zou betekenen dat $C$ $<C<C>>$ moet accepteren(, volgens de definitie van $C$).
    \[ C \text{ verwerpt } <C> \quad\Rightarrow\quad B \text{ verwerpt } <C<C>> \quad\Rightarrow\quad C \text{ accepteert } <C> \]
    Bijgevolg kan $B$ niet bestaan.
    Contradictie.
  \end{proof}
\end{st}

\begin{gev}
  \label{gev:a-tm-niet-coherk}
  $A_{TM}$ is niet co-herkenbaar.
  \begin{proof}
    $A_{TM}$ is herkenbaar, maar niet beslisbaar, dus $A_{TM}$ is niet co-herkenbaar.\stref{st:herk-plus-coherk-is-besl}
  \end{proof}
\end{gev}

\subsection{Het halting probleem}

\begin{de}
  \label{de:h-tm}
  $H_{TM}$ is de taal van koppels $(M,s)$ waarbij $M$ een Turingmachine is die stopt bij input $s$.
  \[ H_{TM} = \{ <M,s> \ |\ M \text{ is een Turinmgmachine en } M \text{ stopt bij input } s \} \]
\end{de}

\begin{st}
  \label{st:h-tm-herk}
  $H_{TM}$ is herkenbaar.
  \begin{proof}
    De herkenner $H$ voor $H_{TM}$ laat bij input $<M,s>$ $M$ lopen op $s$.
    Als $M$ $s$ stopt, dan accepteert $H$ $<M,s>$.
  \end{proof}
\end{st}

\begin{st}
  \label{st:h-tm-niet-besl}
  $H_{TM}$ is niet beslisbaar.

  \begin{proof}
    Stel dat $H_{TM}$ beslisbaar is door een beslisser $H$.
    We construeren nu een beslisser $B$ voor $A_{TM}$ vanuit $H$.
    Bij input $<M,s>$ doet laat $B$ eerst $H$ lopen op $<M,s>$.
    Als $H$ accepteert(, dus $M$ stopt bij input $s$) laat $B$ $M$ lopen op $s$ en geeft als resultaat het resultaat van $M$.
    Als $H$ niet accepteert(, dus $M$ stopt niet bij input $s$) verwerpt $B$ ook.
    Vermits er geen beslisser bestaat voor $A_{TM}$, kan $H$ niet bestaan en is $H_{TM}$ dus ook niet beslisbaar.
  \end{proof}
\end{st}

\begin{gev}
  \label{gev:h-tm-niet-coherk}
  $H_{TM}$ is niet co-herkenbaar.
  \begin{proof}
    $H_{TM}$ is herkenbaar, maar niet beslisbaar, dus $H_{TM}$ is niet co-herkenbaar.\stref{st:herk-plus-coherk-is-besl}
  \end{proof}
\end{gev}

\subsection{Het equivalentieprobleem}

\begin{de}
  \label{de:eq-tm}
  $EQ_{TM}$ is de taal van alle koppels Turingmachines die dezelfde taal herkennen.
\end{de}

\begin{st}
  \label{st:eq-tm-besl}
  $EQ_{TM}$ is niet beslisbaar.

  \begin{proof}
    Het is niet beslisbaar of een TM geen enkele string aanvaardt.
    't Is te zeggen of de taal herkend door een TM gelijk is aan de lege taal.
    Het is dus zeker niet beslisbaar of twee TM's dezelfde taal herkennen.
  \end{proof}
\end{st}

\begin{st}
  \label{st:eq-tm-niet-herk}
  $EQ_{TM}$ is niet herkenbaar.

  \begin{proof}
    We construeren een reductie $f$ van $A_{TM}$ naar $\overline{EQ_{TM}}$.
    Vermits $\overline{A_{TM}}$ niet herkenbaar is, is $EQ_{TM}$ ook niet herkenbaar.\stref{st:reduceerbaarheid-taal-complement}
    Definieer $M_{s}$ als een machine die elke string accepteert als $M$ de string $s$ accepteert en $M_{\phi}$ als de Turingmachine die de lege taal bepaalt.
    \[ f: <M,s> \mapsto <M_{s},M_{\phi}> \]
    \begin{itemize}
    \item $f$ is Turing-berekenbaar. \clarify{waarom? Hoe constreren we de encodering van $M_{s}$ en $M_{\phi}$??}
    \item Als $M$ de string $s$ accepteert, dan zijn $M_{s}$ en $M_{\phi}$ verschillend.
      $M_{s}$ accepteert elke string wanneer $M$ $s$ accepteert terwijl $M_{\phi}$ geen enkele string accepteert.
    \item Als $m$ de string $s$ niet accepteert, dan zijn $M_{s}$ en $M_{\phi}$ gelijk.
      Als $M$ $s$ accepteert accepteren $M_{s}$ en $M_{\phi}$ beide geen enkele string.
    \end{itemize}
  \end{proof}
\end{st}

\begin{st}
  \label{st:eq-tm-niet-coherk}
  $EQ_{TM}$ is niet co-herkenbaar.

  \begin{proof}
    We constreren een reductie $f$ van $A_{TM}$ naar $EQ_{TM}$.
    Vermits $\overline{A_{TM}}$ niet herkenbaar is, is $\overline{EQ_{TM}}$ ook niet herkenbaar.\stref{st:reduceerbaarheid-taal-complement}
    Definieer $M_{s}$ als een machine die elke string accepteert als $M$ de string $s$ accepteert en $M_{\Sigma^{*}}$ als de Turingmachine die elke string aanvaardt.
    \[ f:\ <M,s> \mapsto <M_{s},M_{\Sigma^{*}}> \]
    \begin{itemize}
    \item $f$ is Turing-berekenbaar. \clarify{waarom? Hoe constreren we de encodering van $M_{s}$ en $M_{\Sigma^{*}}$??}
    \item Als $M$ $s$ accepteert, dan zijn $M_{s}$ en $M_{\Sigma^{*}}$ gelijk.
      $M_{s}$ en $M_{\Sigma^{*}}$ accepteren dan beide elke string.
    \item Als $M$ $s$ niet accepteert, dan zijn $M_{s}$ en $M_{\Sigma^{*}}$ verschillend.
      Als $M$ $s$ niet accepteert, accepteert $M_{s}$ immers geen enkele string terwijl $M_{\Sigma^{*}}$ elke string accepteert.
    \end{itemize}
  \end{proof}
\end{st}

\subsection{Het leegheidprobleem}


\begin{de}
  \label{de:e-tm}
  $E_{TM}$ is de taal van alle Turingmachines die geen enkele string aanvaardt.
  \[ E_{TM} = \{ <M,s>\ |\ M \text{ is een Turingmachine en } L_{M} = \phi \} \]
\end{de}

\begin{st}
  \label{st:e-tm-niet-besl}
  $E_{TM}$ is niet beslisbaar.

  \begin{proof}
    Bewijs uit het ongerijmde\\
    Stel dat $E_{TM}$ beslisbaar is, dan bestaat er een beslisser $E$ voor $E_{TM}$.
    Vanuit $E$ construeren we nu een beslisser $B$ voor $A_{TM}$ als volgt:
    $B$ krijgt als invoer een string van de vorm $<M,s>$ en moet beslissen of $M$ de string $s$ aanvaardt.
    \begin{itemize}
    \item $B$ construeert een machine $M_{s}$ die voor invoer $w$ $w$ verwerpt wanneer $w$ verschillend is van $s$ en $M$ laat lopen op $w$ als $w$ gelijk is aan $s$ en het resultaat terug geeft.
    \item $B$ laat $E$ lopen op $<M_{s}>$
      \begin{itemize}
      \item Als $E$ $<M_{s}>$ accepteert, dan verwerpt $B$ $<M,s>$, want dan accepteert $M_{s}$ geen enkele string (ook niet $s$).
      \item Als $E$ $<M_{s}>$ accepteert, dan verwerpt $B$ $<M,s>$ want dan accepteert $M_{s}$ minstens \'e\'en string (enkel $s$).
      \end{itemize}
    \end{itemize}
    Omdat $A_{TM}$ niet beslisbaar is\stref{st:a-tm-niet-besl}, kan $E_{TM}$ ook niet beslisbaar zijn.
  \end{proof}
\end{st}

\subsection{Het regulieretaalprobleem}


\begin{de}
  \label{de:regular-tm}
  $REGULAR_{TM}$ is de taal van alle Turingmachines die een reguliere taal bepaalt.
\end{de}

\begin{st}
  \label{st:regular-tm-niet-besl}
  $REGULAR_{TM}$ is niet beslisbaar.

  \begin{proof}
    Bewijs uit het ongerijmde\\
    Stel dat $REGULAR_{TM}$ beslisbaar is, dan bestaat er een beslisser $R$ voor $REGULAR_{TM}$.
    Vanuit $R$ construeren we nu een beslisser $B$ voor $A_{TM}$ als volgt:
    $B$ krijgt als invoer een string van de vorm $<M,s>$ en moet beslissen of $M$ de string $s$ aanvaardt.
    \begin{itemize}
    \item $B$ construeert een hulpmachine $H_{s}$ die bij invoer $x$ het volgende doet:
      \begin{itemize}
      \item Als $x$ van de vorm $0^{n}1^{n}$ is, dan accepteert $H_{s}$ $x$.
      \item Als $x$ niet van de vorm $0^{n}1^{n}$ is, dan laat $H$ $M$ lopen op $s$ en geeft het resultaat terug.
      \end{itemize}
    \item $B$ laat $R$ lopen op $<H_{s}>$ en geeft het resultaat terug.
    \end{itemize}
    Merk op dat $H_{s}$ ofwel $0^{n}1^{n}$ (niet-regulier\vbref{vb:0n-1n-niet-regulier}) accepteert (wanneer $M$ $s$ niet accepteert), ofwel heel $\Sigma^{*}$ (regulier) accepteert (wanneer $M$ $s$ accepteert).
    $B$ accepteert $<M,s>$ als en slechts als $R$ $<H_{s}>$ accepteert.
    Dit is precies wanneer $H_{s}$ heel $\Sigma^{*}$ accepteert en $M$ dus $s$ accepteert.
    $B$ is bijgevolg een beslisser voor $A_{TM}$.
    Omdat $A_{TM}$ niet beslisbaar is\stref{st:a-tm-niet-besl}, kan $E_{TM}$ ook niet beslisbaar zijn.
  \end{proof}
\end{st}


\section{Weetjes}
\label{sec:weetjes}

\begin{st}
  \[ RegLan \subsetneq DCFL \subsetneq CFL \subsetneq Besl \subsetneq Herk \subsetneq \mathcal{P}(\Sigma^{*}) \]
  \extra{bewijs}
\end{st}

\begin{st}
  $DCFL$ is gesloten onder complement, maar $CFL$ niet. 
\extra{bewijs}
\end{st}

\begin{st}
  $Besl$ is gesloten onder de unie en het complement.
\extra{bewijs}
\end{st}

\begin{st}
  $Herk$ is gesloten onder de unie.
\extra{bewijs}
\end{st}

\begin{st}
  $RegLan$, $CFL$ en $Herk$ zijn gesloten onder de omkering.
\extra{bewijs}
\end{st}

\begin{st}
  $DCFL$ is niet gesloten onder de omkering.
\extra{bewijs zie p 109 voor vb}
\end{st}

\begin{de}
  Een verzameling $V$ is \term{effectief aftelbaar} of \term{opsombaar} als en een TM bestaat die alle elementen uit $v$ (eventueel geencodeerd) \'e\'en voor \'e\'en genereert.
\end{de}

\section{Algemene problemen}
\label{sec:algemene-problemen}

\begin{de}
  Acceptatie\\
  $A_{\phi}$ is de taal van alle $\phi$ $(M,s)$ waarbij $M\in \phi$ een machine is die $s$ aanvaardt.
  \[ A_{\phi} = \{ <M,s> \ |\ M\in \phi \wedge s \in L_{M} \} \]
\end{de}

\begin{de}
  Stopt\\
  $H_{\phi}$ is de taal van alle $\phi$ $(M,s)$ waarbij $M\in \phi$ een machine is die stopt bij input $s$.
  \[ H_{\phi} = \{ <M,s> \ |\ M \in \phi \wedge M \text{ stopt bij input } s \} \]
\end{de}

\begin{de}
  Leegheid\\
  $E_{\phi}$ is de taal van alle $\phi$ $M$ die de lege taal bepalen.
  \[ E_{\phi} = \{ <M,s>\ |\ M \in \phi \wedge L_{M} = \emptyset \} \]
\end{de}

\begin{de}
  Gelijkheid\\
  $EQ_{\phi}$ is de taal van alle $\phi$ $M_{1},M_{2}$ waarbij $M_{1}$ en $M_{2}$ dezelfde taal bepalen.
  \[ EQ_{\phi} = \{ <M_{1},M_{2}> \ |\ M_{1},M_{2} \in \phi \wedge L_{M_{1}} = L_{M_{2}} \} \]
\end{de}

\begin{de}
  Lege string\\
  $ES_{\phi}$ is de taal van alle $\phi$ $M$ die de lege string aanvaarden.
  \[ ES_{\phi} = \{ <M> \ |\ M\in \phi \wedge \epsilon \in L_{M}\} \]
\end{de}

\begin{de}
  Regulier\\
  $REGULAR_{\phi}$ is de taal van alle $\phi$ $M$ die een reguliere taal bepalen.
\end{de}

\begin{de}
  $ALL_{\phi}$ is de taal van alle $\phi$ $M$ die alle strings aanvaarden.
\end{de}

\begin{de}
  Eindigheid\\
  $FINITE_{\phi}$ is de taal van alle $\phi$ $M$ die een eindig aantal strings aanvaarden.
\end{de}


\begin{figure}[!p]  
  \centering
  \begin{tabular}[H]{cc}
    \pvak{\footnotesize{Herkenbaar}}{\footnotesize{Niet Co-herkenbaar}}{H} & \pvak{\footnotesize{Niet Herkenbaar}}{\footnotesize{Co-herkenbaar}}{C}
  \end{tabular}
  \begin{tabular}[H]{cccc}
    \pvak{\ref{de:a-re}}{$A_{RE}$}{B} & \pvak{\ref{de:a-cfg}}{$A_{CFG}$}{B} & \pvak{\ref{de:a-csg}}{$A_{CSG}$}{B} & \pvak{\ref{de:a-tm}}{$A_{TM}$}{H}\\
    \pvak{\ref{de:h-re}}{$H_{RE}$}{B} & \pvak{\ref{de:h-cfg}}{$H_{CFG}$}{B} & \pvak{\ref{de:h-csg}}{$H_{CSG}$}{B} & \pvak{\ref{de:h-tm}}{$H_{TM}$}{H}\\
    \pvak{\ref{de:e-re}}{$E_{RE}$}{B} & \pvak{\ref{de:e-cfg}}{$E_{CFG}$}{B} & \pvak{\ref{de:e-csg}}{$E_{CSG}$}{H} & \pvak{\ref{de:e-tm}}{$E_{TM}$}{C}\\
    \pvak{\ref{de:eq-re}}{$EQ_{RE}$}{B} & \pvak{\ref{de:eq-cfg}}{$EQ_{CFG}$}{H} & \pvak{\ref{de:eq-csg}}{$EQ_{CSG}$}{H} & \pvak{\ref{de:eq-tm}}{$EQ_{TM}$}{N}\\
    \pvak{\ref{de:es-re}}{$ES_{RE}$}{B} & \pvak{\ref{de:es-cfg}}{$ES_{CFG}$}{B} & \pvak{\ref{de:es-csg}}{$ES_{CSG}$}{B} & \pvak{\ref{de:es-tm}}{$ES_{TM}$}{H}\\
    \pvak{\ref{de:regular-re}}{$REGULAR_{RE}$}{B} & \pvak{\ref{de:regular-cfg}}{$REGULAR_{CFG}$}{N} & \pvak{\ref{de:regular-csg}}{$REGULAR_{CSG}$}{N} & \pvak{\ref{de:regular-tm}}{$REGULAR_{TM}$}{N}\\
    \pvak{\ref{de:all-re}}{$ALL_{RE}$}{B} & \pvak{\ref{de:all-cfg}}{$ALL_{CFG}$}{H} & \pvak{\ref{de:all-csg}}{$ALL_{CSG}$}{H} & \pvak{\ref{de:all-tm}}{$ALL_{TM}$}{N}\\
    \pvak{\ref{de:finite-re}}{$FINITE_{RE}$}{B} & \pvak{\ref{de:finite-cfg}}{$FINITE_{CFG}$}{B} & \pvak{\ref{de:finite-csg}}{$FINITE_{CSG}$}{N} & \pvak{\ref{de:finite-tm}}{$FINITE_{TM}$}{N}\\
  \end{tabular}
  \caption{Algemene problemen}
  \label{fig:algemene-problemen}
\end{figure}


\end{document}