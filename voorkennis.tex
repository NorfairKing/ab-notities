\documentclass[main.tex]{subfiles}
\begin{document}

\chapter{Voorkennis}
\label{cha:voorkennis}

\section{Verzamelingenleer}
\label{sec:verzamelingenleer}

\begin{de}
  Een wiskundige verzameling $V$ is een verzameling van verschillende ongeordende elementen $e_1,e_2,\ldots,e_3$.
  \[ \{V = e_1,e_2,\ldots,e_n\} \]
  We noteren ``$e_i$ is een element van $V$'' als volgt.
  \[ e_i \in V \]
\end{de}

\begin{de}
  We noemen $V'$ een deelverzameling van $V$ als elk element van $V'$ ook een element van $V$ is.
  \[ \forall v' \in V' \Rightarrow v \in V \]
\end{de}

\begin{de}
  De machtsverzameling $\mathcal{P}(V)$ van een verzameling $V$ is de verzameling van alle deelverzamelingen van die verzameling.
  \[ \{v\ |\ v\subset V \}\] 
\end{de}

\section{Abstracte algebra}
\label{sec:abstracte-algebra}

\begin{de}
  Een \emph{algebra} is een verzameling $V$ waarop een aantal inwendige operaties $\circ,\ldots$ gedefinieerd zijn.
  \[ \forall\ v_1,v_2,\ldots,v_n \in V:\ z = \circ(v_1,v_2,\ldots,v_n) \in V \]
\end{de}


\end{document}