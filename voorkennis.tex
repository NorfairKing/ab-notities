\documentclass[main.tex]{subfiles}
\begin{document}

\chapter{Voorkennis}
\label{cha:voorkennis}

\section{Verzamelingenleer}
\label{sec:verzamelingenleer}

\begin{de}
  Een wiskundige verzameling $V$ is een verzameling van verschillende ongeordende elementen $e_1,e_2,\ldots,e_3$.
  \[ \{V = e_1,e_2,\ldots,e_n\} \]
  We noteren ``$e_i$ is een element van $V$'' als volgt.
  \[ e_i \in V \]
\end{de}

\begin{de}
  We noemen $V'$ een deelverzameling van $V$ als elk element van $V'$ ook een element van $V$ is.
  \[ \forall v' \in V' \Rightarrow v \in V \]
\end{de}

\begin{de}
  De machtsverzameling $\mathcal{P}(V)$ van een verzameling $V$ is de verzameling van alle deelverzamelingen van die verzameling.
  \[ \{v\ |\ v\subset V \}\] 
\end{de}

\begin{de}
  \label{de:partitie}
  Een partitie $P$ van een verzameling $X$ is een deelverzameling van de machtsverzameling $\mathcal {P}(x)$ van $X$ met de volgende eigenschappen:
  \begin{itemize}
  \item De verzamelingen zijn niet leeg.
    \[ \forall A \in P:\ A \neq \emptyset \]
  \item De verzamelingen zijn onderling disjunct.
    \[ \forall A,B \in P:\ A \neq B \Rightarrow A \cap B = \emptyset \]
  \item De verzamelingen samen vormen $X$.
    \[ \forall x \in X:\ \exists A \in P:\ x \in A \]
  \end{itemize}
\end{de}

\begin{st}
  \label{st:verband-partitie-equivalentierelatie}
  Elke partitie komt overeen met een Equivalentierelatie. Zij $P$ een partitie van $X$.
  De volgende verzameling vormt dan een equivalentierelatie op $X$.
  \[ x \sim y \Leftrightarrow (\exists A \in P:\ x \in A \wedge y \in A )\]

  \begin{proof}
    We definieren een relatie $\sim$ als volgt:
    \[
    x \sim y \Leftrightarrow \text{ x en y zitten in dezelfde deelverzameling van } P
    \]
    Dat deze relatie een equivalentierelatie is volgt meteen uit het feit dat ``dezelfde ... als'' ook een equivalentierelatie is.
  \end{proof}
\end{st}

\section{Abstracte algebra}
\label{sec:abstracte-algebra}

\begin{de}
  Een \term{algebra} is een verzameling $V$ waarop een aantal inwendige operaties $\circ,\ldots$ gedefinieerd zijn.
  \[ \forall\ v_1,v_2,\ldots,v_n \in V:\ z = \circ(v_1,v_2,\ldots,v_n) \in V \]
\end{de}

\begin{de}
  Een \term{subalgebra} is een deelverzameling van een algebra, die op zichzelf ook een algebra vormt met dezelfde operaties.
\end{de}

\end{document}