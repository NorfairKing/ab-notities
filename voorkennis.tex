\documentclass[main.tex]{subfiles}
\begin{document}

\chapter{Voorkennis}
\label{cha:voorkennis}

\section{Verzamelingenleer}
\label{sec:verzamelingenleer}

\begin{de}
  Een wiskundige verzameling $V$ is een verzameling van verschillende ongeordende elementen $e_1,e_2,\ldots,e_3$.
  \[ \{V = e_1,e_2,\ldots,e_n\} \]
  We noteren ``$e_i$ is een element van $V$'' als volgt.
  \[ e_i \in V \]
\end{de}

\begin{de}
  We noemen $V'$ een deelverzameling van $V$ als elk element van $V'$ ook een element van $V$ is.
  \[ \forall v' \in V' \Rightarrow v \in V \]
\end{de}

\begin{de}
  De machtsverzameling $\mathcal{P}(V)$ van een verzameling $V$ is de verzameling van alle deelverzamelingen van die verzameling.
  \[ \{v\ |\ v\subset V \}\] 
\end{de}

\begin{de}
  \label{de:partitie}
  Een partitie $P$ van een verzameling $X$ is een deelverzameling van de machtsverzameling $\mathcal {P}(x)$ van $X$ met de volgende eigenschappen:
  \begin{itemize}
  \item De verzamelingen zijn niet leeg.
    \[ \forall A \in P:\ A \neq \emptyset \]
  \item De verzamelingen zijn onderling disjunct.
    \[ \forall A,B \in P:\ A \neq B \Rightarrow A \cap B = \emptyset \]
  \item De verzamelingen samen vormen $X$.
    \[ \forall x \in X:\ \exists A \in P:\ x \in A \]
  \end{itemize}
\end{de}

\begin{st}
  \label{st:verband-partitie-equivalentierelatie}
  Elke partitie komt overeen met een Equivalentierelatie. Zij $P$ een partitie van $X$.
  De volgende verzameling vormt dan een equivalentierelatie op $X$.
  \[ x \sim y \Leftrightarrow (\exists A \in P:\ x \in A \wedge y \in A )\]

  \begin{proof}
    We definieren een relatie $\sim$ als volgt:
    \[
    x \sim y \Leftrightarrow \text{ x en y zitten in dezelfde deelverzameling van } P
    \]
    Dat deze relatie een equivalentierelatie is volgt meteen uit het feit dat ``dezelfde ... als'' ook een equivalentierelatie is.
  \end{proof}
\end{st}

\begin{de}
  We noemen een partitie $P_{1}$ \term{fijner} dan een andere partitie $P_{2}$ als elk element van $P_{1}$ gelijk is aan, of in een element zit van $P_{2}$.
  Het omgekeerde van fijner is \term{groffer}.
\end{de}

\section{Relaties}
\label{sec:relaties}

\begin{de}
  De \term{transitieve sluiting} $R^{+}$ van een binaire relatie $R$ op een verzameling $M$ is de kleinste transitieve relatie op $M$ die de oorspronkelijke relatie omvat.
  \[ xR^{+}y \Leftrightarrow \exists x_{1},\dotsc,x_{n-1}:\ xRx_{1} \wedge x_{1}Rx_{2} \wedge \dotsb \wedge x_{n-1} R y \]
\end{de}

\subsection{Equivalentierelaties}
\extra{equivalentierelaties}


\subsection{Orderelaties}

\begin{de}
  Een relatie $R$ op een verzameling $X \times X$ is \term{anti-symmetrisch} als het volgende geldt:
  \[ \forall x,y \in X: ((x,y) \in R \wedge (y,x) \in R) \Rightarrow x = y \]
\end{de}

\begin{de}
  Een (parti\"ele) \term{orderelatie} op $X$ is reflexief, transitief en anti-symmetrisch.
\end{de}

\begin{de}
  Een \term{grootste element} $a$ van een verzameling $A$ waarop een orderelatie $\prec$ is gedefinieerd is, is het element waarvoor geldt dan alle andere elementen kleiner zijn of gelijk aan $a$.
  \[ \forall x \in A: x \preceq a \] 
  Analoog wordt ook een \term{kleinste element} gedefinieerd.
\end{de}

\begin{de}
  Een \term{maximaal element} $a$ van $A$ waarop een orderelatie $\prec$ is gedefinieerd is, is het element waarvoor geldt dat er geen kleiner bestaat.
  \[ \not\exists x \in A: a \prec x \]
  Analoog wordt ook een \term{minimaal element} gedefinieerd.
\end{de}

\begin{opm}
  Een maximaal/minimaal element is niet noodzakelijk een grootste/kleinste element.    
\end{opm}

\begin{de}
  Zij $(X,\preceq)$ een geordende verzameling en $A \subsetneq X$.
  $b \in X$ is een \term{bovengrens} van $A$ als het volgende geldt.
  \[ \forall x \in A: x \preceq b \]
  Analoog wordt een \term{ondergrens} gedefinieerd.
\end{de}

\begin{opm}
  Een grens van een ordeverzameling hoeft dus niet in die verzameling te zitten.
\end{opm}

\begin{de}
  Een \term{supremum}(\term{infimum}) van een deelverzameling van een geordende verzameling is een bovengrens(ondergrens) die kleiner(groter) is dan elke andere bovengrens(ondergrens).
\end{de}

\begin{opm}
  Een supremum/infimum is een grens van een ordeverzameling en hoeft dus niet in die verzameling te zitten.
\end{opm}

\section{Abstracte algebra}
\label{sec:abstracte-algebra}

\begin{de}
  Een \term{algebra} is een verzameling $V$ waarop een aantal inwendige operaties $\circ,\ldots$ gedefinieerd zijn.
  \[ \forall\ v_1,v_2,\ldots,v_n \in V:\ z = \circ(v_1,v_2,\ldots,v_n) \in V \]
\end{de}

\begin{de}
  Een \term{subalgebra} is een deelverzameling van een algebra, die op zichzelf ook een algebra vormt met dezelfde operaties.
\end{de}

\begin{de}
  Een \term{tralie} is een verzameling $V$ waarop een partiele orderelatie $\prec$ gedefinieerd is.
  Elke twee elementen in een tralie hebben bovendien een supremum en een infimum.
\end{de}

\end{document}