\documentclass[main.tex]{subfiles}
\begin{document}
\chapter{Berekenbaarheid}
\label{cha:berekenbaarheid}


\section{Post Correspondence}
\label{sec:post-correspondence}

\begin{de}
  Een \term{(domino)steen} over een alfabet $\Sigma$ is een koppel strings $(a,b)$ over $\Sigma$.
  Het eerste deel van het koppel noemen we de bovenkant en het tweede deel de onderkant.
\end{de}

\begin{de}
  \label{post:correspondence}
  Een \term{Post correspondence} spel.\\
  Gegeven een eindige verzameling stenen $S$ is de vraag:
  \begin{center}
    Bestaat er een opeenvolging van stenen uit $S$, zodat de string in de bovenkanten samen gelijk is aan de string in de onderkanten samen.
  \end{center}
\end{de}

\begin{de}
  \label{de:turing-compleet}
  Een formalisme is \term{Turing-compleet} als het de berekeningen van een Turingmachine kan nabootsen.
\end{de}

\begin{st}
  \label{st:pc-is-turing-compleet}
  Het Post correspondence spel is Turing-compleet.
\extra{bewijs, bekijk vb in cursus p 114}
\end{st}

\begin{gev}
  Het Post correspondence spel is onbeslisbaar voor verzamelingen stenen over een alfabet groter dan $1$ symbool.
\extra{bewijs}
\end{gev}

\extra{2-tag systems}


\section{Recursieve functies}
\label{sec:recursieve-functies}

\subsection{Primitief recursieve functies}
\label{sec:prim-recurs-funct}

\subsubsection{Basisfuncties}
\label{sec:basisfuncties}

\begin{de}
  De \term{nulfunctie} is een functie die elk natuurlijk getal afbeeldt op nul.
  \[ nul:\ \mathbb{N} \rightarrow \mathbb{N}:\ x \mapsto 0 \]
\end{de}

\begin{de}
  De \term{successorfunctie} is een functie die elk natuurlijk getal afbeeldt op het volgende natuurlijk getal.
  \[ succ: \mathbb{N} \rightarrow \mathbb{N}:\ x \mapsto x+1 \]
\end{de}

\begin{de}
  De $i$-de \term{projectie} van een $n$-tal is een functie die een $n$-tal afbeeldt op het $i$-de element uit dat $n$-tal.
  \[ p_{i}^{n}: \mathbb{N}^{n} \rightarrow \mathbb{N}:\ (x_{1},x_{2}, \dotsc, x_{n}) \mapsto x_{i} \]
\end{de}

\subsubsection{Compositie}
\label{sec:compositie}

\begin{de}
  De \term{compositiefunctie} $Cn$ is een functie die een functie $f$ 'evalueert', gegeven zijn argumenten.
  Gegeven de volgende elementen:
  \begin{itemize}
  \item 
    $m$ functies $g_{i}$: $g_{1},g_{2},\dotsc,g_{m}$ met elk $k$ argumenten.
    \[ g_{i}:\ \mathbb{N}^{k} \rightarrow \mathbb{N} \]
  \item 
    $f$, een functie met $m$ argumenten.
    \[ f:\ \mathbb{N}^{m} \rightarrow \mathbb{N} \]
  \end{itemize}
  We kunnen dan een $h$ construeren als volgt:\footnote{$\bar{x} = (x_{1},x_{2},\dotsc,x_{k})$}
  \[ h:\ \mathbb{N}^{k} \rightarrow \mathbb{N}:\ \bar{x} \mapsto f(g_{1}(\bar{x}),g_{2}(\bar{x}),\dotsc,g_{m}(\bar{x})) \]
  We noteren $h$ als volgt:
  \[ h = Cn(f,g_{1},\dotsc,g_{m}) \]
\end{de}

\subsubsection{Primitieve recursie}
\label{sec:primitieve-recursie}

\begin{de}
  De \term{primitieve recursiefunctie} is een functie die toelaat om functies recursief samen te stellen.
  Gegeven de volgende elementen...
  \begin{itemize}
  \item Een functie $f$ die een $k$-tal natuurlijke getallen afbeeldt op een natuurlijk getal.
    \[ f: \mathbb{N}^{k} \rightarrow \mathbb{N} \]
  \item Een functie $g$ die een $k+2$-tal natuurlijke getallen afbeeldt op een natuurlijk getal.
    \[ g: \mathbb{N}^{k+2} \rightarrow \mathbb{N} \]
  \end{itemize}
  ... kunnen we een functie $h$ construeren die recursief te werk gaat.
  \[ 
  h:\ \mathbb{N}^{k+1} \rightarrow \mathbb{N}:\ 
  \left\{
    \begin{array}{ll}
      h(\bar{x},0) &= f(\bar{x})\\
      h(\bar{x},y) &= g(\bar{x}, y-1, h(\bar{x},y-1))\\
    \end{array}
  \right.
  \]
  We noteren $h$ tenslotte als volgt:
  \[ h = Pr(f,g) \]
\end{de}

\begin{de}
  Een \term{primitief recursieve functie} is een functie die je kan maken door functies samen te stellen met behulp van de compositie en primitieve recursie.
\end{de}

\subsection{Een niet-primitief recursieve functie}
\label{sec:een-niet-primitief}

\begin{de}
  De \term{Ackerman functie} is als volgt gedefinieerd:
  \[ 
  Ack:\ \mathbb{N}^{2} \rightarrow \mathbb{N}:\ 
  \left\{
    \begin{array}{llll}
      Ack(0,y)   &= y+1\\
      Ack(x,0)   &= Ack(x-1,1)\\
      Ack(x,y)   &= Ack(x-1,Ack(x,y-1))\\
    \end{array}
  \right.
  \]
\end{de}

\begin{st}
  De Ackerman functie is niet primitief recursief. \zb
\end{st}

\begin{de}
  De \term{onbegrensde minimisatie} functie is een proces dat voor een gegeven functie de minimale nulpunten vindt.\\
  Gegeven een functie $f: \mathbb{N}^{k+1} \rightarrow \mathbb{N}$ construeren we de onbegrensde minimisatie als volgt:
  \[
  g:\ \mathbb{N}^{k} \rightarrow \mathbb{N}:\ g(\bar{x}) = y
  \Leftrightarrow
  \left\{
    \begin{array}{rl}
      f(\bar{x},y) &= 0\\
      \forall z \in \mathbb{N}: z < y &\Rightarrow  f(\bar{x},z) \neq 0
    \end{array}
  \right.
  \]
  We noteren $g$ als volgt:
  \[ g = Mn(f) \]
\end{de}

\begin{opm}
  Een eenvoudigere manier om je $Mn$ voor te stellen ziet er als volgt uit:
  \begin{verbatim}
    y = 0;
    while (f(x,y) != 0)
      y++;
    return y;
  \end{verbatim}
\end{opm}

\begin{de}
  Een \term{recursieve functie} is een functie die je verkrijgt door het samenstellen van de basisfuncties, de primitieve recursie, de samenstelling en de onbegrensde minimisatie.
\end{de}

\begin{st}
  Elke recursieve functie is Turing-berekenbaar en omgekeerd. \zb
\end{st}

\section{De busy beaver functie}
\label{sec:de-busy-beaver}

\begin{de}
  Een \term{busy beaver} Turingmachine is de Turingmachine (over het alfabet $\{0,1\}$) die bij een lege input het meeste enen op de band zet van alle Turingmachines met $n$ toestanden en toch stopt.
\end{de}

\begin{de}
  Het aantal enen op de band gezet door een busy beaver Turingmachine met $n$ toestanden noteren we als $\Sigma(n)$.
  \[ \Sigma:\ \mathbb{N} \rightarrow \mathbb{N}  \]
\end{de}

\begin{st}
  $\Sigma$ is niet Turingberekenbaar.

  \begin{proof}
    De enige manier om na te gaan of een Turingmachine met $n$ staten een zogenaamede busy beaver is, is door de machine te laten lopen en te tellen hoeveel enen die produceert.
    \waarom
    Dit aantal moet dan worden vergeleken met de uitvoer van de andere Turingmachines met evenveel staten om te concluderen of de machine al dan niet een Turingmachine is.
    Het probleem ligt erin dat het niet beslisbaar (en dus ook niet co-herkenbaar) is of een Turingmachine al dan niet zal stoppen voor een gegeven (lege) invoer.
    Er valt dus ook niet te beslissen welke van de Turingmachines met $n$ staten het meeste enen zal schrijven en toch nog stoppen.
  \end{proof}
\end{st}

\begin{st}
  $\Sigma$ stijgt sneller dan elke Turingberekenbare functie. \zb
\end{st}

\end{document}