\message{ !name(symbolen-strings-talen.tex)}\documentclass[main.tex]{subfiles}
\begin{document}

\message{ !name(symbolen-strings-talen.tex) !offset(10) }
\underline{eindige} verzameling van symbolen.
\end{de}

\begin{de}
  Een \term{string} $s$ over een alfabet $\Sigma$ is een geordende opeenvolging van nul, \'e\'en of meer elementen van $\Sigma$. Symbolen zijn dus strings van lengte 1.
  \[ s = a_{1}\ldots a_{n} \text{ met } a_{i} \in \Sigma \]
\end{de}

\begin{de}
  $\epsilon$ is de \term{string} zonder symbolen en noemen we de \term{lege string}.
\end{de}

\begin{opm}
  $\epsilon$ is een notatie voor `niets'. Het symbool wordt slechts gebruikt omdat effectief `niets' (ook niet dat woord) opschrijven onpractisch is.
\end{opm}

\begin{opm}
  Bij twijfel kan elk symbool gezien worden als een string van lengte $1$.
\end{opm}

\begin{de}
  De \term{concatenatie} $xy$ van twee strings $x = \{x_1,x_2,\ldots,x_m\}$ en $y =   \{y_1,y_2,\ldots,y_n\}$ is de volgende geordende verzameling  
  \[
  xy = \{ x_1,x_2,\ldots,x_m,y_1,y_2,\ldots,y_n\}
  \] 
\message{ !name(symbolen-strings-talen.tex) !offset(373) }

\end{document}
