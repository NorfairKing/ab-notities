\documentclass[main.tex]{subfiles}
\begin{document}

\chapter{Symbolen, Strings en Talen}
\label{cha:symbolen-strings-talen}

\section{Symbolen en Strings}
\label{sec:symbolen-en-strings}

\begin{de}
  Een \term{symbool} $s$ is een representatie van \textit{een object} in de abstractste zin van het woord.
  De enige voorwaarde voor een symbool is dat er een equivalentierelatie (de gelijkheid) op gedefinieerd is met symbolen van dezelfde soort.
\end{de}

\begin{de}
  Een alfabet $\Sigma$ is een \textit{eindige} verzameling van symbolen.
\end{de}

\begin{de}
  Een \term{string} $s$ over een alfabet $\Sigma$ is een geordende opeenvolging van nul, \'e\'en of meer elementen van $\Sigma$.
  \[ s = a_{1}\ldots a_{n} \quad\text{ met }\quad a_{i} \in \Sigma \]
  Symbolen zijn dus strings van lengte 1.
\end{de}

\begin{de}
  $\epsilon$ is de \textit{string} zonder symbolen en noemen we de \term{lege string}.
\end{de}

\begin{opm}
  $\epsilon$ is een notatie voor `niets'. Het symbool wordt slechts gebruikt omdat effectief `niets' (ook niet dat woord) opschrijven onpractisch is.
\end{opm}

\begin{de}
  De \term{concatenatie} $xy$ van twee strings $x = \{x_1,x_2,\ldots,x_m\}$ en $y =   \{y_1,y_2,\ldots,y_n\}$ is de volgende geordende opeenvolging.
  \[
  xy = x_1x_2\ldots x_my_1y_2\ldots y_n
  \] 
\end{de}

\begin{ei}
  De concatenatie van strings is associatief:
  \[
  (xy)z = x(yz)
  \]

  \begin{proof}
    \[
    \begin{array}{r l l}
      (xy)z &= \{x_1,x_2,\ldots,x_m,y_1,y_2,\ldots,y_n\}z &\\
            &= \{x_1,x_2,\ldots,x_m,y_1,y_2,\ldots,y_n,z_1,z_2,\ldots,z_o\} &\\
            &= x\{y_1,y_2,\ldots,y_n,z_1,z_2,\ldots,z_o\} &= x(yz)
    \end{array}
    \]
  \end{proof}
\end{ei}

\begin{de} 
  De verzameling van alle eindige strings over een alfabet $\Sigma$ noteren we als $\Sigma^{*}$.
  \[ \Sigma^{*} = \{ a_{1}a_{2}\ldots a_{n}\ |\ a_{i}\in \Sigma,\ n,i\in \mathbb{N} \} \]
\end{de}

\begin{de}
  De verzameling $\Sigma \cup \{\epsilon\}$ noteren we korter als $\Sigma_{\epsilon}$.
  Merk op dat dit niet zomaar een verzameling symbolen is. $\epsilon$ is namelijk een string.
\end{de}

\begin{de}
  De \term{omgekeerde string} $s^{R}$ van een string $s$ is de string waarbij de symbolen van $s$ in omgekeerde volgorde staan.
  \[ s = a_{n}\ldots a_{1} \text{ met } a_{i} \in \Sigma \]
\end{de}

\section{Talen}
\label{sec:talen}

\begin{de}
  Een \term{taal} $L$ over een alfabet $\Sigma$ is een verzameling van eindige strings over $\Sigma$.
\end{de}

\begin{de}
  De \term{concatenatie} $L_1L_2$ \term{van twee talen} $L_1$ en $L_2$ over hetzelfde alfabet $\Sigma$ is de volgende verzameling:
  \[
  L_1L_2 = \{\ xy\ |\ x \in L_1,\ y \in L_2\ \} 
  \]
\end{de}

\begin{ei}
  De concatenatie van talen is associatief:
  \[
  (L_1L_2)L_3 = L_1(L_2L_3)
  \]

  \begin{proof}
    \[
    \begin{array}{r l l}
      (L_1L_2)L_3 &= \{\ xy\ |\ x \in L_1,\ y \in L_2\ \}L_3 &\\
                 &= \{\ xyz\ |\ x \in L_1,\ y \in L_2\,\ z \in L_3\} &\\
                 &= L_1\{\ yz\ |\ y \in L_2,\ z \in L_3\ \} &= L_1(L_2L_3)
    \end{array}
    \]
  \end{proof}
\end{ei}

\begin{ei}
  Talen, uitgerust met de unie, de doorsnede, het complement en de concatenatie, vormen een algebra.
  \begin{proof}
    Inderdaad, zowel de unie, de doorsnede, het complement als de concatenatie zijn inwendig. 
  \end{proof}
\end{ei}

\begin{de}
  De concatenatie van $n$ keer een taal $L$ met zichzelf noteren we als $L^n$.
  $L^0$ bevat enkel de lege string.
  \[
  L^0 = \{\epsilon\},\quad L^{n} = LL^{n-1}
  \]
\end{de}

\begin{de}
  De \term{Kleene-ster} $L^*$ van een taal $L$ is de unie van alle concatenaties van $L$ met zichzelf.
  \[
  L^* = \bigcup_{n=0}^{\infty}L^n
  \]
\end{de}

\begin{de}
  $L^{+}$ is de unie van $L$, \'e\'en of meer keer geconcateneerd met zichzelf.
  \[
  L^{+} = LL^{*}
  \]
\end{de}

\begin{ei}
  \label{ei:taal-alternatieve-definitie}
  We kunnen een taal ook defini\"eren als een deelverzameling van $\Sigma^{*}$ (of als een element van $\mathcal{P}(\Sigma^{*})$.)

  \begin{proof}
    Inderdaad, elke verzameling van eindige strings is een deelverzameling van de verzameling van alle eindige strings, alsook een element van de verzameling van alle deelverzamelingen van de verzameling van alle eindige strings.
  \end{proof}
\end{ei}

\begin{de}
  $L_{\Sigma}$ is de notatie voor de verzameling van alle \term{talen over een alfabet} $\Sigma$. 
  \[ L_{\Sigma} = \mathcal{P}(\Sigma^{*}) \]
\end{de}

\section{Reguliere expressies}
\label{sec:reguliere-expressies}

\begin{de}
  Een \term{reguliere expressie} (RE) over een alfabet $\Sigma$ wordt inductief gedefinieerd als een expressie van de volgende vorm:
  \begin{itemize}
  \item $\epsilon$
  \item $\phi$
  \item $a$ met $a \in \Sigma$
  \item $(E_1E_2)$ waarbij $E_1$ en $E_2$ reguliere expressies zijn over $\Sigma$
  \item $(E)^*$ waarbij $E$ een reguliere expressie is over $\Sigma$
  \item $(E_1|E_2)$ waarbij $E_1$ en $E_2$ reguliere expressies zijn over $\Sigma$
  \end{itemize}
  Merk op dat, net als strings, reguliere expressies steeds eindig zijn.
  Deze definitie laat immers geen oneindige reguliere expressies toe.
\end{de}

\begin{de}
  De verzameling van alle reguliere expressies over een alfabet $\Sigma$ noteren we als $RegEx_{\Sigma}$ of $RegEx$ als het alfabet duidelijk is.
\end{de}

\begin{st}
  De verzameling van alle reguliere expressies $RegEx_{\Sigma}$ over een alfabet $\Sigma$ vormt een taal.

  \begin{proof}
    Inderdaad, voeg aan $\Sigma$ nog de volgende symbolen toe om $\Sigma'$ te bekomen: $\epsilon$, $\phi$, $($, $)$, $|$ en $^{*}$.
    Nu vormt de verzameling van alle reguliere expressies een taal over $\Sigma'$.

    Merk op dat deze taal \emph{niet} regulier\deref{de:reguliere-taal} is. Er zitten immers haakjes in die moeten samen passen.\stref{st:an-bn-niet-regulier-pompend-lemma} \stref{st:an-bn-niet-regulier-mn-relatie}
  \end{proof}
\end{st}

\begin{de}
  \label{def:taal-bepaald-door-regex}
  De \term{taal bepaald door een reguliere expressie} $L_E$ over hetzelfde alfabet $\Sigma$ is de volgende.
  \[
  \begin{array}{|c|c|}
    \hline
    E                           & L_E\\
    \hline
    a \text{ met } a \in \Sigma & \{a\}\\
    \epsilon                    & {\epsilon}\\
    \phi                        & \emptyset\\
    (E_1E_2)                    & L_{E1}L_{E2}\\
    (E)^{*}                      & L_E^*\\
    (E_1|E_2)                   & L_{E1} \cup L_{E2}\\
    \hline
  \end{array}
  \]
\end{de}

\begin{de}
  \label{de:reguliere-taal}
  Een \term{reguliere taal} is een taal die bepaald wordt door een reguliere expressie.
\end{de}

\begin{ei}
  \label{ei:reguliere-taal-expressie}
  Voor elke reguliere taal bestaat er een reguliere expressie die die taal bepaalt.

  \begin{proof}
    Inderdaad, anders was het geen reguliere taal! \deref{de:reguliere-taal}
  \end{proof}
\end{ei}

\begin{st}
  Als een reguliere expressie $E$ geen ster bevat, dan is de taal $L_{E}$ bepaald door die reguliere expressie eindig.
  
  \begin{proof}
    We tonen eerst de kardinaliteit van van $L_{E}$ afhankelijk van $E$.
    \begin{figure}[H]
      \centering
      \[
      \begin{array}{|c|c|}
        \hline
        E                           & |L_E|\\
        \hline
        a \text{ met } a \in \Sigma & 1\\
        \epsilon                    & 1\\
        \phi                        & 0\\
        (E_1E_2)                    & |L_{E1}| \cdot |L_{E2}|\\
        (E)^{*}                      & \infty\\
        (E_1|E_2)                   & |L_{E1}| + |L_{E2}| - |L_{E1} \cap L_{E2}|\\
        \hline
      \end{array}
      \]
      \caption{Kardinaliteit van de taal bepaald door een regex}
      \label{fig:regex-cardinaliteiten}
    \end{figure}   
    \noindent Zoals we zien in de tabel blijft de kardinaliteit $L_{E}$ eindig zolang we geen ster gebruiken in $E$.
  \end{proof}
\end{st}

\begin{opm}
  De omgekeerde stelling geldt niet.
  Een reguliere expressie met een ster bepaalt niet noodzakelijk een oneindige taal.
  Tegenvoorbeeld:
  \[ L_{\epsilon^{*}} \text{ bevat precies \'e\'en element } \]
\end{opm}

\begin{st}
  Zij $E$ en $F$ reguliere expressies. Nu geldt volgende bewering.
  \[ L_{E} \subseteq L_{(E|F)} \]

  \begin{proof}
    \[ L_{(E|F)} = L_{E} \cup L_{F}\]
    \[ L_{E} \subseteq L_{E} \cup L_{F} \]
  \end{proof}
\end{st}

\begin{de}
  $RegLan$ is de \term{verzameling van alle reguliere talen}.
\end{de}

\begin{ei}
  $Reglan$ is een subalgebra van $L_{\Sigma}$.

  \begin{proof}
    Bewijs in delen.
    \begin{itemize}
    \item $RegLan$ is een deelverzameling van $L_{\Sigma}$.
      \[ RegLan \subseteq L_{\Sigma} \]
    \item De unie is inwendig in $RegLan$.\\
      Kies twee willekeurige reguliere talen $L_{E1},\ L_{E2} \in RegLan$.
      De unie $L_{E1} \cup L_{E2}$ van deze twee talen wordt bepaald door de reguliere expressie $(E_1|E_2)$ en is bijgevolg een reguliere taal.
      \deref{def:taal-bepaald-door-regex}
    \item De concatenatie is inwendig in $RegLan$.\\
      Kies twee willekeurige reguliere talen $L_{E1},\ L_{E2} \in RegLan$.
      De concatenatie $L_{E1}L_{E2}$ van deze twee talen wordt bepaald door de reguliere expressie $E_1E_2$ en is bijgevolg een reguliere taal.
      \deref{def:taal-bepaald-door-regex}
    \item Het complement is inwendig in $RegLan$.
      \TODO{Verwijzing naar bewijs (later)}
    \item De doorsnede is inwendig in $RegLan$.
      De doorsnede van twee reguliere talen $L_{1}$ en $L_{2}$ valt te schrijven als volgt.
      \[ L_{1} \cap L_{2} = \overline{\overline{L_{1}} \cup \overline{L_{2}}} \]
      Omdat zowel de unie van twee reguliere talen als het complement van twee reguliere talen een  reguliere taal is, is ook de doorsnede van twee reguliere talen een reguliere taal.
    \end{itemize}
  \end{proof}
\end{ei}

\begin{ei}
  Hier volgen een aantal eigenschappen over $RegLan$.
  \begin{enumerate}
  \item $RegLan \subseteq \Sigma$: Onwaar, er wordt hier een verzameling talen met een verzameling symbolen vergeleken.
  \item $RegLan \subseteq \Sigma^{*}$: Onwaar, er wordt hier een verzameling talen met een verzameling strings vergeleken.
  \item $RegLan \subseteq \mathcal P(\Sigma)$: Onwaar, er wordt hier een verzameling talen met een verzameling van verzamelingen van symbolen vergeleken.
  \item $RegLan \subseteq \mathcal P(\Sigma^{*})$: Waar, dit is equivalent met: ``Een reguliere taal is een taal.''.
  \item $RegLan \subseteq \mathcal P(\mathcal P(\Sigma^{*}))$: Onwaar, er wordt hier een verzameling talen vergeleken met een een verzameling van verzamelingen van talen. Merk op dat, als er '$\in$' stond in plaats van '$\subseteq$', deze stelling wel klopte.
  \item $(\forall x)(x \in RegLan \Rightarrow x \in \Sigma)$: Onwaar, zie puntje 1.
  \item $(\forall x)(x \in RegLan \Rightarrow x \in \Sigma^{*})$: Onwaar, zie puntje 2.
  \item $(\forall x)(x \in RegLan \Rightarrow x \in \mathcal P(\Sigma))$: Onwaar, zie puntje 3.
  \item $(\forall x)(x \in RegLan \Rightarrow x \in \mathcal P(\Sigma^{*}))$: Waar, zie puntje 4.
  \item $(\forall x)(x \in RegLan \Rightarrow x \in \mathcal P(\mathcal P(\Sigma^{*})))$: Onwaar, zie puntje 5.
  \item $(\forall x,y)(x \in RegLan \wedge y \in x \Rightarrow y \in \Sigma)$: Onwaar, $y$ is hier een string terwijl $x$ een taal is. De uitdrukking $y \in \Sigma$ is dus triviaal fout.
  \item $(\forall x,y)(x \in RegLan \wedge y \in x \Rightarrow y \in \Sigma^{*})$: Waar, dit is equivalent met: ``Een string van een reguliere taal is een string.''.
  \item $(\forall x,y)(x \in RegLan \wedge y \in x \Rightarrow y \in \mathcal P(\Sigma)$: Onwaar, $y$ is een string, maar $\mathcal P(\Sigma)$ is een verzameling van verzamelingen symbolen. Er is op deze twee geen 'element van' gedefinieerd.
  \item $(\forall x,y)(x \in RegLan \wedge y \in x \Rightarrow y \in \mathcal P(\Sigma^{*})$: Onwaar, $y$ is een string, maar $\mathcal P(\Sigma^{*})$ is een verzameling talen. Er is op deze twee geen 'element van' gedefinieerd.
  \item $(\forall x,y)(x \in RegLan \wedge y \in x \Rightarrow y \in \mathcal P(\mathcal P(\Sigma^{*}))$: Onwaar, $y$ is een string, maar $\mathcal P(\mathcal P(\Sigma^{*}))$ is een verzameling van verzamelingen van talen. Er is op deze twee geen 'element van' gedefinieerd.
  \end{enumerate}
\end{ei}

\begin{st}
  Elke eindige taal $L$ is regulier.

  \begin{proof}
    We bewijzen dit door de constructie van een reguliere expressie die $L$ bepaalt:\\
    Zij $n$ het aantal strings in $L$ met $n$ eindig.
    Voor elke string $s_{i} \in L$, construeren we een reguliere expressie $E_{i}$ die de taal met enkel $s_{i}$ bepaalt door de concatenatie van de opeenvolgende symbolen in $s_{i}$. 
    Voeg nu al deze reguliere expressies $E_{i}$ samen tot $(E_{1}|E_{2}|\ldots|E_{n})$ om de reguliere expressie te krijgen die $L$ bepaalt.
  \end{proof}
\end{st}

\begin{st}
  Zij $L$ een reguliere taal en $s \not \in L$ een string.
  $L' = L \cup \{s\}$ is regulier.
  \begin{proof}
    $L$ is een reguliere taal, dus er bestaat een reguliere expressie $E$ die $L$ bepaalt.\deref{de:reguliere-taal}
    \[ L = L_{E} \]
    Beschouw nu de reguliere expressie $E'$.
    \[ E' = (E | s) \]
    $E'$ is nu een reguliere expressie voor $L'$. 
    $L'$ is dus regulier.
  \end{proof}
\end{st}

\begin{st}
  \label{sigma-ster-aftelbaar}
  Oneindige talen over een alfabet $\Sigma$ zijn aftelbaar.

  \begin{proof}
    Beschouw het alfabet $\Sigma$ als de symbolen van een $|\Sigma|$-tallig talstelsel.
    Elke string $s$ in $\Sigma^{*}$ komt nu overeen met een getal $n_{s} \in N$.
    Er bestaat dus een bijectie tussen $\mathbb N$ en $\Sigma^{*}$. 
    Nu is $\Sigma^{*}$ aftelbaar oneindig omdat $N$ aftelbaar oneindig is.
  \end{proof}
\end{st}

\begin{st}
  Elke oneindige reguliere taal $L$ is aftelbaar in aantal elementen.

  \begin{proof}
    $L$ is een deelverzameling van $\Sigma^{*}$\eiref{ei:taal-alternatieve-definitie}, dus $L$ is hoogstens aftelbaar oneindig.
  \end{proof}
\end{st}

\begin{st}
  \label{st:overaftelbaar-veel-talen}
  Er zijn overaftelbaar veel talen over een alfabet $\Sigma$.

  \begin{proof}
    De talen $L_{\Sigma}$ over een alfabet zijn allemaal een deelverzameling van $\Sigma^{*}$. \eiref{ei:taal-alternatieve-definitie} $L_{\Sigma}$ is dus de machtsverzameling van $\Sigma^{*}$.
    Bovendien is de machtsverzameling van een aftelbare verzameling\stref{sigma-ster-aftelbaar} overaftelbaar.
    Er zijn dus overaftelbaar oneindig veel talen over een alfabet $\Sigma$.
  \end{proof}
\end{st}

\begin{st}
  Er bestaat een niet-reguliere taal.

  \begin{proof}
    Elke reguliere expressie bepaalt precies \'e\'en (reguliere) taal.
    Er zijn aftelbaar oneindig veel reguliere expressies en bijgevolg hoogstens aftelbaar oneindig veel reguliere talen.
    Er zijn echter overaftelbaar oneindig veel talen.\stref{st:overaftelbaar-veel-talen}
    Er moet dus minstens \'e\'en niet-reguliere taal bestaan.
    (In feite zijn de meeste\footnote{'de meeste' heeft een heel specifieke wiskundige betekenis.} talen niet-regulier.)
  \end{proof}
\end{st}

\begin{de}
  De omgekeerde taal $L^{R}$ is de waarin alle strings van $L$ omgekeerd zijn is regulier.
  \[ L^{R} = \{ s^{R}\ |\ s \in L \} \]
\end{de}

\begin{st}
  \label{omgekeerde-reguliere-taal-is-regulier}
  Voor elke reguliere taal $L$ is de omgekeerde taal $L^{R}$ regulier.

  \begin{proof}
    Zij $L$ een willekeurige reguliere taal, dan bestaat er een reguliere expressie $E$ die $L$ bepaalt.
    We construeren nu de reguliere expressie $E'$ die $L^{R}$ bepaalt.
    
    Voor elke mogelijke vorm van $E$ bestaat er een overeenkomstige $E'$ die recursief geconstrueerd kan worden.
    \[
    \begin{array}{|c|c|}
      \hline
      E                           & E'\\
      \hline
      a \text{ met } a \in \Sigma & a \text{ met } a \in \Sigma\\
      \epsilon                    & \epsilon\\
      \phi                        & \phi\\
      (E_1E_2)                    & (E_{2}'E_{1}')\\
      (E)^{*}                      & (E'^{*})\\
      (E_1|E_2)                   & (E_1'|E_2')\\
      \hline
    \end{array}
    \]
    Zoals we zien wordt eigenlijk enkel het concatenatie geval aangepast. 
  \end{proof}
\end{st}

\end{document}
